

\INEchaptercarta{Inversión pública en SAN}{}

\addtocounter{Cuadro}{1}
%%%%%%%%%%%%%%%%%%


\hoja{
	{\Bold\Large 7.1	Integración del presupuesto en seguridad alimentaria y nutricional del Plan del Pacto Hambre Cero a nivel nacional. }\\
	\normalsize
	{\Bold\color{color1!80!black}\parbox{15cm}{Cuadro \theCuadro $\,-$ Presupuesto en Seguridad Alimentaria y Nutricional (SAN), del Plan del Pacto Hambre Cero (PH0), por criterios de seguimiento y porcentaje de ejecución; según institución.}}\\
	{\Bold\color{color1!80!black}{República de Guatemala, año 2012.}}\\
	{\color{color1!80!black}{(Quetzales y porcentaje)}}\\[-0.3cm]
	\begin{center}\fontsize{3.0mm}{1.3em}\selectfont \setlength{\arrayrulewidth}{0.7pt}
		\begin{tabular}{x{4.0cm}rrrc}
			\hline &&&&\\[-0.4cm]  
			\multicolumn{1}{x{2.0cm}}{\raisebox{-.3cm}{\textbf{Institución}}} &	\multicolumn{3}{x{6.3cm}}{\textbf{Criterios de seguimiento (en quetzales)}}&\multicolumn{1}{x{2cm}}{\multirow{2}{*}[-.5mm]{\textbf{Ejecución}}}\\[0.05cm]\cline{2-4}
			\multicolumn{1}{l}{ }&\multicolumn{4}{l}{ }\\[-.35cm]
			\multicolumn{1}{x{2.0cm}}{ } &	\multicolumn{1}{x{2cm}}{\textbf{Asignado}}&\multicolumn{1}{x{2cm}}{\textbf{Vigente}}&\multicolumn{1}{x{2cm}}{\textbf{Ejecutado}}&\multicolumn{1}{x{2cm}}{\textbf{(Porcentaje)}}\\[0.05cm]\hline
			\rowcolor{color1!40!white} \multicolumn{1}{l}{\Bold{	Total presupuesto	 }}& 	4,733,725,081	&	5,601,097,205	&	4,995,879,003	&	89.2	\\
			\rowcolor{color1!20!white} \multicolumn{1}{l}{\Bold{	Ministerios	 }}& 	4,008,928,485	&	5,028,707,801	&	4,526,928,708	&	90.0	\\
			\multicolumn{1}{l}{	Ministerio de Educación Pública -Mineduc-	 }& 	638,253,918	&	560,143,362	&	558,267,850	&	99.7	\\
			\rowcolor{color1!5!white}\multicolumn{1}{l}{	Ministerio de Salud Pública y Asistencia Social -Mspas-	 }& 	590,752,798	&	669,883,381	&	552,014,678	&	82.4	\\
			\multicolumn{1}{l}{	Ministerio de Economía -Mineco- 	 }& 	56,469,411	&	58,456,590	&	54,607,485	&	93.4	\\
			\rowcolor{color1!5!white}\multicolumn{1}{l}{	Ministerio de Trabajo -Mintrab-	 }& 	10,344,362	&	3,181,159	&	2,898,509	&	91.1	\\
			\multicolumn{1}{l}{	Ministerio de Agricultura, Ganadería y Alimentación -MAGA-	 }& 	1,180,732,672	&	696,660,339	&	647,975,496	&	93.0	\\
			\rowcolor{color1!5!white}\multicolumn{1}{l}{	Ministerio de Comunicaciones, Infraestructura y Vivienda  -Micivi-	 }& 	1,445,807,023	&	1,896,224,766	&	1,671,325,521	&	88.1	\\
			\multicolumn{1}{l}{	Ministerio de Ambiente y Recursos Naturales  -MARN-	 }& 	86,568,301	&	51,592,577	&	48,916,964	&	94.8	\\
			\rowcolor{color1!5!white}\multicolumn{1}{l}{	Ministerio de Desarrollo Social -Mides-	 }& 	0	&	1,092,565,628	&	990,922,206	&	90.7	\\
			\rowcolor{color1!20!white} \multicolumn{1}{l}{\Bold{	Secretarías	 }}& 	449,454,765	&	299,213,569	&	288,137,447	&	96.3	\\
			\multicolumn{1}{l}{	Secretaría de Coordinación Ejecutiva de la Presidencia -SCEP-	 }& 	45,954,770	&	44,450,835	&	42,821,670	&	96.3	\\
			\rowcolor{color1!5!white}\multicolumn{1}{l}{	Secretaría de Bienestar Social -SBS-	 }& 	204,945,078	&	114,626,067	&	107,713,856	&	94.0	\\
			\multicolumn{1}{l}{	Secretaría de Obras Sociales de la Esposa del Presidente -Sosep-	 }& 	155,892,084	&	102,139,270	&	100,071,030	&	98.0	\\
			\rowcolor{color1!5!white}\multicolumn{1}{l}{	Secretaría de Seguridad Aliementaria y Nutricional -Sesan-	 }& 	42,662,833	&	37,997,397	&	37,530,890	&	98.8	\\
			\rowcolor{color1!20!white} \multicolumn{1}{l}{\Bold{	Descentralizadas	 }}& 	275,341,831	&	273,175,835	&	180,812,849	&	66.2	\\
			\multicolumn{1}{l}{	Coordinadora Nacional para la Reducción de Desastres -Conred-	 }& 	8,466,582	&	6,300,586	&	6,162,996	&	97.8	\\
			\rowcolor{color1!5!white}\multicolumn{1}{l}{	Instituto de Ciencia y Tecnología Agrícola -ICTA-	 }& 	42,000,000	&	42,000,000	&	28,170,923	&	67.1	\\
			\multicolumn{1}{l}{	Comisión Nacional de Alfabetización -Conalfa-	 }& 	204,875,249	&	204,875,249	&	137,433,878	&	67.1	\\
			\rowcolor{color1!5!white}\multicolumn{1}{l}{	Instituto Nacional de Comercialización Agrícola -Indeca-	 }& 	20,000,000	&	20,000,000	&	9,045,052	&	45.2	\\[0.05cm]
			\hline
			&&&\\[-0.36cm]\end{tabular}\addtocounter{Cuadro}{1}
	\end{center}
	{\footnotesize Fuente:  Elaborado por Sesan con datos de Sicoin-Minfin, 2016.}\\[.1cm]
	%	{\parbox{13cm}{\footnotesize \textbf{Notas:} Se denomina desnutrición severa cuando los niños están 3 desviaciones estándar o más por debajo de la media, de acuerdo a la tabla de medidas de la OMS. }}\\[.3cm]
	%	{\parbox{13cm}{\footnotesize Se denomina desnutrición total cuando los niños están dos desviaciones estándar o más por debajo de la media. Incluye a los niños que están 3 desviaciones estándar o más por debajo de la media.}}\\
}



%%%%%%%%%%%%%%%%%%%%%%

%%%%%%%%%%%%%%%%%%


\hoja{
	\normalsize
	{\Bold\color{color1!80!black}{Cuadro \theCuadro $\,-$Presupuesto en Seguridad Alimentaria y Nutricional (SAN), del Pacto Hambre Cero (PH0);}}\\
	{\Bold\color{color1!80!black}{ por criterios de seguimiento y porcentaje de ejecución, según institución}}\\
	{\Bold\color{color1!80!black}{República de Guatemala, año 2013.}}\\
	{\color{color1!80!black}{(Quetzales y porcentaje)}}\\[-0.3cm]
	\begin{center}\fontsize{3.0mm}{1.3em}\selectfont \setlength{\arrayrulewidth}{0.7pt}
		\begin{tabular}{x{5.5cm}rrc}
			\hline &&&\\[-0.4cm]  
			\multicolumn{1}{x{5.5cm}}{\raisebox{-.3cm}{\textbf{Instituciones}}} &	\multicolumn{2}{x{4.6cm}}{\textbf{Criterios de seguimiento (en quetzales)}}&\multicolumn{1}{x{2cm}}{\multirow{2}{*}[-.5mm]{\textbf{Ejecución}}}\\[0.05cm]\cline{2-3}
			\multicolumn{1}{l}{ }&\multicolumn{3}{l}{ }\\[-.35cm]
			\multicolumn{1}{c}{ }&\multicolumn{1}{x{2cm}}{\textbf{Vigente}}&\multicolumn{1}{x{2cm}}{\textbf{Ejecutado}}&\multicolumn{1}{x{2cm}}{\textbf{(Porcentaje)}}\\[0.05cm]\hline
			\rowcolor{color1!40!white} \multicolumn{1}{l}{\Bold{	Total presupuesto	 }}& 	6,099,516,355	&	4,575,679,524	&	75.0	\\
			\rowcolor{color1!20!white} \multicolumn{1}{l}{\Bold{	Ministerios	 }}& 	5,444,656,944	&	4,300,095,736	&	79.0	\\
			\multicolumn{1}{l}{	Ministerio de Educación Pública -Mineduc-	 }& 	737,101,217	&	584,016,414	&	79.2	\\
			\rowcolor{color1!5!white}\multicolumn{1}{l}{	Ministerio de Salud Pública y Asistencia Social -Mspas-	 }& 	948,552,777	&	875,150,770	&	92.3	\\
			\multicolumn{1}{l}{	Ministerio de Economía -Mineco-	 }& 	9,420,886	&	9,420,885	&	100.0	\\
			\rowcolor{color1!5!white}\multicolumn{1}{l}{	Ministerio de Trabajo -Mintrab- 	 }& 	24,770,815	&	24,117,799	&	97.4	\\
			\multicolumn{1}{l}{	Ministerio de Agricultura Ganadería y Alimentación -MAGA-	 }& 	1,054,903,855	&	632,593,189	&	60.0	\\
			\rowcolor{color1!5!white}\multicolumn{1}{l}{	Ministerio de Comunicaciones, Infraestructura y Vivienda -Micivi-	 }& 	1,611,284,421	&	1,423,438,995	&	88.3	\\
			\multicolumn{1}{l}{	Ministerio de Ambiente y Recursos Naturales -MARN-	 }& 	1,304,562	&	914,439	&	70.1	\\
			\rowcolor{color1!5!white}\multicolumn{1}{l}{	Ministerio de Desarrollo Social -Midex-	 }& 	1,057,318,411	&	750,443,245	&	71.0	\\
			\rowcolor{color1!20!white} \multicolumn{1}{l}{\Bold{	Secretarías	 }}& 	130,448,713	&	104,688,447	&	80.3	\\
			\multicolumn{1}{l}{	Secretaría de Coordinación Ejecutiva de la Presidencia -SCEP-	 }& 	473,205	&	395,608	&	83.6	\\
			\rowcolor{color1!5!white}\multicolumn{1}{l}{	Secretaría de Bienestar Social -SBS-	 }& 	2,003,362	&	1,965,564	&	98.1	\\
			\multicolumn{1}{l}{	Secretaría de Obras Sociales de la Esposa del Presidente -Sosep-	 }& 	34,548,326	&	34,548,325	&	100.0	\\
			\rowcolor{color1!5!white}\multicolumn{1}{l}{	Secretaria Seprem	 }& 	823,184	&	822,361	&	99.9	\\
			\multicolumn{1}{l}{	Sesan	 }& 	92,600,636	&	66,956,590	&	72.3	\\
			\rowcolor{color1!20!white} \multicolumn{1}{l}{\Bold{	Descentralizadas	 }}& 	524,410,699	&	170,895,340	&	32.6	\\
			\multicolumn{1}{l}{	ICTA	 }& 	45,182,076	&	34,292,886	&	75.9	\\
			\rowcolor{color1!5!white}\multicolumn{1}{l}{	Infom	 }& 	399,883,688	&	70,170,245	&	17.5	\\
			\multicolumn{1}{l}{	Conalfa 	 }& 	60,344,935	&	52,208,071	&	86.5	\\
			\rowcolor{color1!5!white}\multicolumn{1}{l}{	Indeca	 }& 	19,000,000	&	14,224,138	&	74.9	\\
			[0.05cm]
			\hline
			&&&\\[-0.36cm]\end{tabular}\addtocounter{Cuadro}{1}
	\end{center}
	{\footnotesize Fuente:  Elaborado por Sesan con datos de Sicoin-Minfin, 2016.}\\[.1cm]
	%	{\parbox{13cm}{\footnotesize \textbf{Notas:} Se denomina desnutrición severa cuando los niños están 3 desviaciones estándar o más por debajo de la media, de acuerdo a la tabla de medidas de la OMS. }}\\[.3cm]
	%	{\parbox{13cm}{\footnotesize Se denomina desnutrición total cuando los niños están dos desviaciones estándar o más por debajo de la media. Incluye a los niños que están 3 desviaciones estándar o más por debajo de la media.}}\\
}






%%%%%%%%%%%%%%%%%%


\hoja{
	\normalsize
	{\Bold\color{color1!80!black}{Cuadro \theCuadro $\,-$ Presupuesto en Seguridad Alimentaria y Nutricional, del Plan del Pacto Hambre Cero (PH0), }}\\
	{\Bold\color{color1!80!black}{por criterios de seguimiento y porcentaje de ejecución; según institución.}}\\
	{\Bold\color{color1!80!black}{República de Guatemala, año 2014.}}\\
	{\color{color1!80!black}{(Quetzales y porcentaje)}}\\[-0.3cm]
	\begin{center}\fontsize{3.0mm}{1.3em}\selectfont \setlength{\arrayrulewidth}{0.7pt}
		\begin{tabular}{x{4.0cm}rrrc}
			\hline &&&&\\[-0.4cm]  
			\multicolumn{1}{x{2.0cm}}{\raisebox{-.3cm}{\textbf{Institución}}} &	\multicolumn{3}{x{6.3cm}}{\textbf{Criterios de seguimiento (en quetzales)}}&\multicolumn{1}{x{2cm}}{\multirow{2}{*}[-.5mm]{\textbf{Ejecución}}}\\[0.05cm]\cline{2-4}
			\multicolumn{1}{l}{ }&\multicolumn{4}{l}{ }\\[-.35cm]
			\multicolumn{1}{x{2.0cm}}{ } &	\multicolumn{1}{x{2cm}}{\textbf{Asignado}}&\multicolumn{1}{x{2cm}}{\textbf{Vigente}}&\multicolumn{1}{x{2cm}}{\textbf{Ejecutado}}&\multicolumn{1}{x{2cm}}{\textbf{(Porcentaje)}}\\[0.05cm]\hline
			\rowcolor{color1!40!white} \multicolumn{1}{l}{\Bold{	Total presupuesto	 }}& 	5,271,613,438	&	6,587,699,083	&	5,615,711,668	&	85.2	\\
			\rowcolor{color1!20!white} \multicolumn{1}{l}{\Bold{	Ministerios 	 }}& 	4,507,844,851	&	5,778,816,812	&	5,175,638,909	&	89.6	\\
			\multicolumn{1}{l}{	Mineduc	 }& 	699,552,593	&	700,740,593	&	618,271,098	&	88.2	\\
			\rowcolor{color1!5!white}\multicolumn{1}{l}{	Mspas	 }& 	663,857,018	&	990,638,317	&	841,129,249	&	84.9	\\
			\multicolumn{1}{l}{	MAGA	 }& 	955,572,082	&	877,715,651	&	790,560,249	&	90.1	\\
			\rowcolor{color1!5!white}\multicolumn{1}{l}{	Micivi	 }& 	1,265,437,403	&	2,133,468,319	&	1,874,645,098	&	87.9	\\
			\multicolumn{1}{l}{	MARN	 }& 	1,304,562	&	1,350,231	&	525,833	&	38.9	\\
			\rowcolor{color1!5!white}\multicolumn{1}{l}{	Mides	 }& 	922,121,193	&	1,074,903,701	&	1,050,507,382	&	97.7	\\
			\rowcolor{color1!20!white} \multicolumn{1}{l}{\Bold{	Secretarías	 }}& 	93,163,866	&	120,720,925	&	100,761,731	&	83.5	\\
			\multicolumn{1}{l}{	SCEP	 }& 	544,200	&	544,200	&	510,836	&	93.9	\\
			\rowcolor{color1!5!white}\multicolumn{1}{l}{	Sosep	 }& 	43,937,153	&	52,030,887	&	44,785,106	&	86.1	\\
			\multicolumn{1}{l}{	Sesan	 }& 	45,977,554	&	65,440,879	&	53,033,906	&	81.0	\\
			\rowcolor{color1!5!white}\multicolumn{1}{l}{	SBS	 }& 	2,704,959	&	2,704,959	&	2,431,882	&	89.9	\\
			\rowcolor{color1!20!white} \multicolumn{1}{l}{\Bold{	Descentralizadas	 }}& 	670,604,721	&	688,161,346	&	339,311,027	&	49.3	\\
			\multicolumn{1}{l}{	ICTA	 }& 	38,000,000	&	40,430,500	&	35,150,813	&	86.9	\\
			\rowcolor{color1!5!white}\multicolumn{1}{l}{	Infom	 }& 	362,182,524	&	405,108,963	&	107,945,244	&	26.6	\\
			\multicolumn{1}{l}{	Conalfa	 }& 	194,688,339	&	145,382,916	&	107,601,793	&	74.0	\\
			\rowcolor{color1!5!white}\multicolumn{1}{l}{	Indeca	 }& 	12,000,000	&	22,300,000	&	14,489,251	&	65.0	\\
			\multicolumn{1}{l}{	Fontierras	 }& 	63,733,858	&	74,938,967	&	74,123,927	&	98.9	\\[0.05cm]
			\hline
			&&&&\\[-0.36cm]\end{tabular}\addtocounter{Cuadro}{1}
	\end{center}
	{\footnotesize Fuente:  Elaborado por Sesan con datos de Sicoin-Minfin, 2016.}\\[.1cm]
	%	{\parbox{13cm}{\footnotesize \textbf{Notas:} Se denomina desnutrición severa cuando los niños están 3 desviaciones estándar o más por debajo de la media, de acuerdo a la tabla de medidas de la OMS. }}\\[.3cm]
	%	{\parbox{13cm}{\footnotesize Se denomina desnutrición total cuando los niños están dos desviaciones estándar o más por debajo de la media. Incluye a los niños que están 3 desviaciones estándar o más por debajo de la media.}}\\
}


%%%%%%%%%%%%%%%%%%%4



\hoja{
	\normalsize
	{\Bold\color{color1!80!black}{Cuadro \theCuadro $\,-$ Presupuesto en Seguridad Alimentaria y Nutricional (SAN), del Pacto Hambre Cero (PH0);}}\\
	{\Bold\color{color1!80!black}{ por criterios de seguimiento y porcentaje de ejecución, según institución. }}\\
	{\Bold\color{color1!80!black}{República de Guatemala, año 2015.}}\\
	{\color{color1!80!black}{(Quetzales y porcentaje)}}\\[-0.3cm]
	\begin{center}\fontsize{3.0mm}{1.3em}\selectfont \setlength{\arrayrulewidth}{0.7pt}
		\begin{tabular}{x{4.0cm}rrrc}
			\hline &&&&\\[-0.4cm]  
			\multicolumn{1}{x{2.0cm}}{\raisebox{-.3cm}{\textbf{Institución}}} &	\multicolumn{3}{x{6.3cm}}{\textbf{Criterios de seguimiento (en quetzales)}}&\multicolumn{1}{x{2cm}}{\multirow{2}{*}[-.5mm]{\textbf{Ejecución}}}\\[0.05cm]\cline{2-4}
			\multicolumn{1}{l}{ }&\multicolumn{4}{l}{ }\\[-.35cm]
			\multicolumn{1}{x{2.0cm}}{ } &	\multicolumn{1}{x{2cm}}{\textbf{Asignado}}&\multicolumn{1}{x{2cm}}{\textbf{Vigente}}&\multicolumn{1}{x{2cm}}{\textbf{Ejecutado}}&\multicolumn{1}{x{2cm}}{\textbf{(Porcentaje)}}\\[0.05cm]\hline
			\rowcolor{color1!40!white} \multicolumn{1}{l}{\Bold{	Total presupuesto	 }}& 	5,433,883,259	&	5,342,538,764	&	3,560,292,421	&	66.6	\\
			\rowcolor{color1!20!white} \multicolumn{1}{l}{\Bold{	Ministerios	 }}& 	4,833,238,682	&	4,687,601,799	&	3,235,010,865	&	69.0	\\
			\multicolumn{1}{l}{	Mineduc	 }& 	733,498,088	&	573,507,563	&	573,099,895	&	99.9	\\
			\rowcolor{color1!5!white}\multicolumn{1}{l}{	Mspas	 }& 	1,225,659,293	&	1,590,091,837	&	1,251,483,963	&	78.7	\\
			\multicolumn{1}{l}{	MAGA	 }& 	735,423,108	&	738,297,821	&	368,724,638	&	49.9	\\
			\rowcolor{color1!5!white}\multicolumn{1}{l}{	Micivi	 }& 	1,315,185,738	&	1,211,547,974	&	613,552,025	&	50.6	\\
			\multicolumn{1}{l}{	MARN	 }& 	6,064,342	&	6,264,563	&	5,616,550	&	89.7	\\
			\rowcolor{color1!5!white}\multicolumn{1}{l}{	Mides	 }& 	817,408,113	&	567,892,041	&	422,533,794	&	74.4	\\
			\rowcolor{color1!20!white} \multicolumn{1}{l}{\Bold{	Secretarías	 }}& 	 90,240,558 	&	 113,289,469 	&	 103,343,636 	&	91.2	\\
			\multicolumn{1}{l}{	SCEP	 }& 	3,068,045	&	2,863,616	&	2,540,717	&	88.7	\\
			\rowcolor{color1!5!white}\multicolumn{1}{l}{	Sosep	 }& 	0	&	54,448,531	&	47,897,317	&	88.0	\\
			\multicolumn{1}{l}{	Sesan	 }& 	85,191,335	&	54,052,253	&	50,980,532	&	94.3	\\
			\rowcolor{color1!5!white}\multicolumn{1}{l}{	SBS	 }& 	1,981,178	&	1,925,069	&	1,925,069	&	100.0	\\
			\rowcolor{color1!20!white} \multicolumn{1}{l}{\Bold{	Descentralizadas	 }}& 	 510,404,019 	&	 541,647,496 	&	 221,937,920 	&	41.0	\\
			\multicolumn{1}{l}{	ICTA	 }& 	37,500,000	&	37,500,000	&	32,331,296	&	86.2	\\
			\rowcolor{color1!5!white}\multicolumn{1}{l}{	Infom	 }& 	208,288,358	&	249,401,014	&	36,765,567	&	14.7	\\
			\multicolumn{1}{l}{	Conalfa	 }& 	174,973,497	&	162,404,318	&	125,585,659	&	77.3	\\
			\rowcolor{color1!5!white}\multicolumn{1}{l}{	Indeca	 }& 	17,000,000	&	19,700,000	&	11,148,774	&	56.6	\\
			\multicolumn{1}{l}{	Fontierras	 }& 	72,642,164	&	72,642,164	&	16,106,625	&	22.2	\\
			[0.05cm]
			\hline
			&&&&\\[-0.36cm]\end{tabular}\addtocounter{Cuadro}{1}
	\end{center}
	{\footnotesize Fuente:  Elaborado por Sesan con datos de Sicoin-Minfin, 2016.}\\[.1cm]
	%	{\parbox{13cm}{\footnotesize \textbf{Notas:} Se denomina desnutrición severa cuando los niños están 3 desviaciones estándar o más por debajo de la media, de acuerdo a la tabla de medidas de la OMS. }}\\[.3cm]
	%	{\parbox{13cm}{\footnotesize Se denomina desnutrición total cuando los niños están dos desviaciones estándar o más por debajo de la media. Incluye a los niños que están 3 desviaciones estándar o más por debajo de la media.}}\\
}





%%%%%%%%%%%%%%%%%%% 5

\newpage
{\Bold\Large 7.2	Seguimiento mensual del gasto del Plan del Pacto Hambre Cero. Años 2013 a 2015}
$\ $\\[1cm]
\fontsize{4mm}{1.9em}\selectfont \setlength{\arrayrulewidth}{01pt}
$\ $\\[-1.8cm]
%	{\Bold\color{color1!80!black}{Cuadro \theCuadro $\,-$  Mujeres embarazadas al momento de la encuesta, que recibieron atención pre natal, por establecimiento o lugar a donde asistieron; según características varias. }}\\
%	{\Bold\color{color1!80!black}{República de Guatemala, año 2008/2009. }}\\
%	\normalsize (Porcentajes)\\[0.4cm]
\begin{center}\fontsize{4mm}{1.8em}
	\selectfont \setlength{\arrayrulewidth}{1pt}
	$\ $\\[-3.5cm]
	$\!$\begin{longtable}{x{4.0cm}rrrc}
		\begin{tabular}{l}\fontsize{3mm}{0.6em}
			\selectfont \setlength{\arrayrulewidth}{1pt}
			$\ $\\[-1.0cm]
			\multicolumn{1}{p{14cm}}{\Bold\color{color1!80!black}{Cuadro \theCuadro $\,-$   Presupuesto en Seguridad Alimentaria y Nutricional (SAN), del Plan del Pacto Hambre Cero (PH0), por criterios de seguimiento asignado, vigente y ejecutado y porcentaje de ejecución; según actividad presupuestaria.  }}\\[-0.09cm]
			\multicolumn{1}{p{14cm}}{\Bold\color{color1!80!black}{República de Guatemala, año 2012. }}\\[-0.1cm]
			\multicolumn{1}{p{14cm}}{\normalsize (Porcentajes)}\\[0.4cm]
		\end{tabular}\\
		%			\fontsize{4mm}{1.8em}
		%			\selectfont \setlength{\arrayrulewidth}{1pt}	
		\hline &&&&\\[-0.4cm]  
		\multicolumn{1}{x{2.0cm}}{\raisebox{-.3cm}{\textbf{Actividad presupuestaria}}} &	\multicolumn{3}{x{6.3cm}}{\textbf{Criterios de seguimiento (en quetzales)}}&\multicolumn{1}{x{2cm}}{\multirow{2}{*}[-.5mm]{\textbf{Ejecución}}}\\[0.05cm]\cline{2-4}
		\multicolumn{1}{l}{ }&\multicolumn{4}{l}{ }\\[-.35cm]
		\multicolumn{1}{x{2.0cm}}{ } &	\multicolumn{1}{x{2cm}}{\textbf{Asignado}}&\multicolumn{1}{x{2cm}}{\textbf{Vigente}}&\multicolumn{1}{x{2cm}}{\textbf{Ejecutado}}&\multicolumn{1}{x{2cm}}{\textbf{(Porcentaje)}}\\[0.05cm]\hline\\[-0.1cm]
		%			&&&&&&&&&&&&&&&& \\[-0.1cm]
		\multicolumn{1}{l}{$\ $} &  \multicolumn{4}{c}{$\ $} \\[-0.48cm]				     
		\hline\endhead
		\hline \multicolumn{5}{r}{\textit{Continúa en la siguiente página}} \\[2cm]
		\endfoot
		%			&&&&\tiny&&&&\tiny&&&&\tiny&&&& \\[-0.3cm]
		\multicolumn{5}{l}{\footnotesize Fuente: Elaborado por Sesan con datos de Sicoin-Minfin, 2016.}\\[-0.1cm]
		%			\multicolumn{5}{l}{\footnotesize * Menos de 25 casos. }\\[-0.1cm]
		\endlastfoot
		\rowcolor{color1!20!white} \multicolumn{1}{p{5.5cm}}{\Bold{	Total presupuesto PPH0	}}&	4,733,725,081	&	5,601,097,205	&	4,995,879,003	&	89.2	\\
		\multicolumn{1}{p{5.5cm}}{	Provisión de alimentación escolar	}&	638,253,918	&	560,143,362	&	558,267,850	&	99.7	\\
		\rowcolor{color1!5!white}\multicolumn{1}{p{5.5cm}}{	Inmunizaciones 	}&	277,051,555	&	294,015,937	&	236,181,729	&	80.3	\\
		\multicolumn{1}{p{5.5cm}}{	Prevención y control de la desnutrición 	}&	45,192,430	&	105,903,807	&	97,396,345	&	92.0	\\
		\rowcolor{color1!5!white}\multicolumn{1}{p{5.5cm}}{	Prevención y promoción de la salud reproductiva 	}&	141,319,770	&	159,183,027	&	112,598,441	&	70.7	\\
		\multicolumn{1}{p{5.5cm}}{	Fortalecimiento a medicamentos (ventana de los 100 días)	}&	78,000,000	&	82,642,733	&	79,234,957	&	95.9	\\
		\rowcolor{color1!5!white}\multicolumn{1}{p{5.5cm}}{	Vigilancia del sistema de salud 	}&	46,822,839	&	24,779,646	&	23,246,682	&	93.8	\\
		\multicolumn{1}{p{5.5cm}}{	Administración de la salud ambiental 	}&	2,366,204	&	3,358,231	&	3,356,524	&	99.9	\\
		\rowcolor{color1!5!white}\multicolumn{1}{p{5.5cm}}{	Desarrollo de la micro, pequeña y mediana empresa	}&	38,135,934	&	41,728,283	&	37,973,903	&	91.0	\\
		\multicolumn{1}{p{5.5cm}}{	Servicios de protección al consumidor	}&	18,333,477	&	16,728,307	&	16,633,582	&	99.4	\\
		\rowcolor{color1!5!white}\multicolumn{1}{p{5.5cm}}{	Regulaciones de asuntos laborales y de empleo 	}&	10,344,362	&	3,181,159	&	2,898,509	&	91.1	\\
		\multicolumn{1}{p{5.5cm}}{	Seguridad alimentaria nutricional	}&	155,818,837	&	58,181,006	&	54,201,097	&	93.2	\\
		\rowcolor{color1!5!white}\multicolumn{1}{p{5.5cm}}{	Disponibilidad de alimentos	}&	91,170,462	&	38,938,703	&	34,972,121	&	89.8	\\
		\multicolumn{1}{p{5.5cm}}{	Desarrollo económico rural agropecuario	}&	904,271,872	&	578,255,366	&	538,698,022	&	93.2	\\
		\rowcolor{color1!5!white}\multicolumn{1}{p{5.5cm}}{	Servicios de coordinación regional y extensión rural	}&	29,471,501	&	21,285,264	&	20,104,255	&	94.5	\\
		\multicolumn{1}{p{5.5cm}}{	Dirección y coordinación desarrollo de la infraestructura vial	}&	67,829,781	&	89,748,090	&	82,554,562	&	92.0	\\
		\rowcolor{color1!5!white}\multicolumn{1}{p{5.5cm}}{	Construcción, ampliación, rehabilitación y pavimentación de carreteras primarias, puentes y distribuidores de tránsito	}&	1,377,977,242	&	1,806,476,676	&	1,588,770,958	&	87.9	\\
		\multicolumn{1}{p{5.5cm}}{	Sistema integrado de gestión ambiental nacional	}&	15,284,512	&	24,472,330	&	22,225,484	&	90.8	\\
		\rowcolor{color1!5!white}\multicolumn{1}{p{5.5cm}}{	Gestión socio-ambiental desconcentrada	}&	26,054,147	&	21,588,533	&	21,288,913	&	98.6	\\
		\multicolumn{1}{p{5.5cm}}{	Conservación y protección de los recursos naturales	}&	32,194,642	&	4,697,990	&	4,695,813	&	100.0	\\
		\rowcolor{color1!5!white}\multicolumn{1}{p{5.5cm}}{	Sustentabilidad ambiental	}&	105,000	&	105,000	&	102,097	&	97.2	\\
		\multicolumn{1}{p{5.5cm}}{	Adaptación y mitigación al cambio climático	}&	12,930,000	&	728,724	&	604,658	&	83.0	\\
		\rowcolor{color1!5!white}\multicolumn{1}{p{5.5cm}}{	Actividades centrales de desarrollo social (Mides)	}&	0	&	878,049,679	&	809,147,283	&	92.2	\\
		\multicolumn{1}{p{5.5cm}}{	Protección al adulto mayor	}&	0	&	25,936,736	&	25,258,180	&	97.4	\\
		\rowcolor{color1!5!white}\multicolumn{1}{p{5.5cm}}{	Hambre cero (Mides)	}&	0	&	133,566,365	&	123,696,743	&	92.6	\\
		\multicolumn{1}{p{5.5cm}}{	Familias seguras	}&	0	&	24,870,407	&	19,321,687	&	77.7	\\
		\rowcolor{color1!5!white}\multicolumn{1}{p{5.5cm}}{	Empleabilidad de jóvenes 	}&	0	&	22,498,890	&	7,711,993	&	34.3	\\
		\multicolumn{1}{p{5.5cm}}{	Productividad rural	}&	0	&	5,699,508	&	5,343,419	&	93.8	\\
		\rowcolor{color1!5!white}\multicolumn{1}{p{5.5cm}}{	Rectoria y coordinación social	}&	0	&	1,944,044	&	442,901	&	22.8	\\
		\multicolumn{1}{p{5.5cm}}{	Coordinación de politicas y proyectos de desarrollo	}&	45,954,770	&	44,450,835	&	42,821,670	&	96.3	\\
		\rowcolor{color1!5!white}\multicolumn{1}{p{5.5cm}}{	Actividadaes de bienestar social 	}&	16,241,606	&	15,418,284	&	15,037,478	&	97.5	\\
		\multicolumn{1}{p{5.5cm}}{	Fortalecimiento y apoyo familiar y comunitario 	}&	98,628,712	&	48,674,625	&	47,360,182	&	97.3	\\
		\rowcolor{color1!5!white}\multicolumn{1}{p{5.5cm}}{	Proteccion, abrigo y rehabilitación familiar	}&	90,074,760	&	50,533,158	&	45,316,195	&	89.7	\\
		\multicolumn{1}{p{5.5cm}}{	Obras sociales (Sosep)	}&	155,892,084	&	102,139,270	&	100,071,030	&	98.0	\\
		\rowcolor{color1!5!white}\multicolumn{1}{p{5.5cm}}{	Coordinación de seguridad alimentaria y nutricional	}&	42,662,833	&	37,997,397	&	37,530,890	&	98.8	\\
		\multicolumn{1}{p{5.5cm}}{	Actividades centrales de gestión de riesgo (Conred)	}&	4,092,582	&	6,184,586	&	6,085,051	&	98.4	\\
		\rowcolor{color1!5!white}\multicolumn{1}{p{5.5cm}}{	Gestión de riesgo 	}&	1,944,000	&	0	&	0	&	0.0	\\
		\multicolumn{1}{p{5.5cm}}{	Emergencia alimentaria en Jalapa	}&	2,430,000	&	116,000	&	77,945	&	67.2	\\
		\rowcolor{color1!5!white}\multicolumn{1}{p{5.5cm}}{	Ciencia, tecnología e innovación agrícola	}&	42,000,000	&	42,000,000	&	28,170,923	&	67.1	\\
		\multicolumn{1}{p{5.5cm}}{	Alfabetización	}&	204,875,249	&	204,875,249	&	137,433,878	&	67.1	\\
		\rowcolor{color1!5!white}\multicolumn{1}{p{5.5cm}}{	Actividades centrales de asistencia alimentaria  (Indeca)	}&	4,302,355	&	4,395,530	&	3,049,136	&	69.4	\\
		\multicolumn{1}{p{5.5cm}}{	Apoyo a la asistencia alimentaria	}&	13,535,328	&	9,442,153	&	5,892,500	&	62.4	\\
		\rowcolor{color1!5!white}\multicolumn{1}{p{5.5cm}}{	Beneficiado de granos básicos	}&	2,162,317	&	6,162,317	&	103,416	&	1.7	\\
		\hline
		&&&&\\[-0.28cm]
	\end{longtable}\addtocounter{Cuadro}{1}
\end{center}




%%%%%%%%%%%%%%%%%%%6

%%%%%%%%%%%%%%%%%%


\hoja{
	\normalsize
	{\Bold\color{color1!80!black}{Cuadro \theCuadro $\,-$Ejecución del Plan del Pacto Hambre Cero (PPH0);según componente y porcentaje de ejecución.}}\\
	{\Bold\color{color1!80!black}{República de Guatemala, año 2013.}}\\
	{\color{color1!80!black}{(Quetzales y porcentaje)}}\\[-0.3cm]
	\begin{center}\fontsize{3.0mm}{1.3em}\selectfont \setlength{\arrayrulewidth}{0.7pt}
		\begin{tabular}{x{5.5cm}rrc}
			\hline &&&\\[-0.4cm]  
			\multicolumn{1}{x{5.5cm}}{\raisebox{-.3cm}{\textbf{Componentes}}} &	\multicolumn{2}{x{4.6cm}}{\textbf{Criterios de seguimiento (en quetzales)}}&\multicolumn{1}{x{2cm}}{\multirow{2}{*}[-.5mm]{\textbf{Ejecución}}}\\[0.05cm]\cline{2-3}
			\multicolumn{1}{l}{ }&\multicolumn{3}{l}{ }\\[-.35cm]
			\multicolumn{1}{c}{ }&\multicolumn{1}{x{2cm}}{\textbf{Vigente}}&\multicolumn{1}{x{2cm}}{\textbf{Ejecutado}}&\multicolumn{1}{x{2cm}}{\textbf{(Porcentaje)}}\\[0.05cm]\hline
			\rowcolor{color1!40!white} \multicolumn{1}{l}{\Bold{	Total presupuesto	}}&	6,099,516,355	&	4,544,224,464	&	74.5	\\
			\multicolumn{1}{l}{	Provisión de servicios básicos de salud y nutrición	}&	947,390,200	&	874,218,707	&	92.3	\\
			\rowcolor{color1!5!white}\multicolumn{1}{l}{	Promoción de lactancia materna y alimentación complementaria	}&	33,783,811	&	33,783,811	&	100.0	\\
			\multicolumn{1}{l}{	Alimentos fortificados	}&	71,166,002	&	69,866,549	&	98.2	\\
			\rowcolor{color1!5!white}\multicolumn{1}{l}{	Atención a población vulnerable a la inseguridad alimentaria	}&	690,718,853	&	503,874,164	&	72.9	\\
			\multicolumn{1}{l}{	Mejoramiento de los ingresos y la economía familia	}&	2,872,537,433	&	2,142,793,380	&	74.6	\\
			\rowcolor{color1!5!white}\multicolumn{1}{l}{	Agua y saneamiento	}&	404,554,198	&	74,182,320	&	18.3	\\
			\multicolumn{1}{l}{	Gobernanza local	}&	34,886,190	&	36,719,499	&	105.3	\\
			\rowcolor{color1!5!white}\multicolumn{1}{l}{	Escuelas saludables	}&	771,997,976	&	603,452,041	&	78.2	\\
			\multicolumn{1}{l}{	Hogar saludable	}&	153,125,923	&	153,125,923	&	100.0	\\
			\rowcolor{color1!5!white}\multicolumn{1}{l}{	Alfabetización	}&	60,344,935	&	52,208,071	&	86.5	\\
			\multicolumn{1}{l}{	Ejes transversales 	}&	59,010,835	&	31,455,060	&	53.3	\\
			[0.05cm]
			\hline
			&&&\\[-0.36cm]\end{tabular}\addtocounter{Cuadro}{1}
	\end{center}
	{\footnotesize Fuente:  Elaborado por Sesan con datos de Sicoin-Minfin, 2016.}\\[.1cm]
	%	{\parbox{13cm}{\footnotesize \textbf{Notas:} Se denomina desnutrición severa cuando los niños están 3 desviaciones estándar o más por debajo de la media, de acuerdo a la tabla de medidas de la OMS. }}\\[.3cm]
	%	{\parbox{13cm}{\footnotesize Se denomina desnutrición total cuando los niños están dos desviaciones estándar o más por debajo de la media. Incluye a los niños que están 3 desviaciones estándar o más por debajo de la media.}}\\
}

%%%%%%%%%%%%%%%%%%%%%%7



\hoja{
	\normalsize
	{\Bold\color{color1!80!black}{Cuadro \theCuadro $\,-$ Ejecución del Plan del Pacto Hambre Cero (PPH0); por criterios de seguimiento y porcentaje de ejecución, según componentes y eje transversal. }}\\
	{\Bold\color{color1!80!black}{República de Guatemala, año 2014.}}\\
	{\color{color1!80!black}{(Quetzales y porcentaje)}}\\[-0.3cm]
	\begin{center}\fontsize{3.0mm}{1.3em}\selectfont \setlength{\arrayrulewidth}{0.7pt}
		\begin{tabular}{x{4.0cm}rrrc}
			\hline &&&&\\[-0.4cm]  
			\multicolumn{1}{x{2.0cm}}{\raisebox{-.3cm}{\textbf{Componente, eje transversal y grupo institucional }}} &	\multicolumn{3}{x{6.3cm}}{\textbf{Criterios de seguimiento (en quetzales)}}&\multicolumn{1}{x{2cm}}{\multirow{2}{*}[-.5mm]{\textbf{Ejecución}}}\\[0.05cm]\cline{2-4}
			\multicolumn{1}{l}{ }&\multicolumn{4}{l}{ }\\[-.35cm]
			\multicolumn{1}{x{2.0cm}}{ } &	\multicolumn{1}{x{2cm}}{\textbf{Asignado}}&\multicolumn{1}{x{2cm}}{\textbf{Vigente}}&\multicolumn{1}{x{2cm}}{\textbf{Ejecutado}}&\multicolumn{1}{x{2cm}}{\textbf{(Porcentaje)}}\\[0.05cm]\hline
			\rowcolor{color1!40!white} \multicolumn{1}{l}{\Bold{	Total presupuesto	}}&	5,249,682,137	&	6,587,699,083	&	5,615,711,668	&	85.2	\\
			\rowcolor{color1!20!white} \multicolumn{1}{l}{\Bold{	Componentes directos	}}&	1,259,691,823	&	1,728,988,117	&	1,545,343,020	&	89.4	\\
			\multicolumn{1}{l}{	Provisión de servicios básicos de salud y nutrición	}&	594,114,453	&	869,376,567	&	736,564,012	&	84.7	\\
			\rowcolor{color1!5!white}\multicolumn{1}{l}{	Promoción de lactancia materna y alimentación complementaria	}&	114,604,369	&	111,669,386	&	89,710,589	&	80.3	\\
			\multicolumn{1}{l}{	Alimentos fortificados	}&	0	&	60,202,921	&	58,046,796	&	96.4	\\
			\rowcolor{color1!5!white}\multicolumn{1}{l}{	Atención a población vulnerable a la inseguridad alimentaria	}&	550,973,001	&	687,739,243	&	661,021,623	&	96.1	\\
			\rowcolor{color1!20!white} \multicolumn{1}{l}{\Bold{	Componente de viabilidad	}}&	3,944,012,760	&	4,793,270,087	&	4,017,334,741	&	83.8	\\
			\multicolumn{1}{l}{	Mejoramiento de los ingresos y la economía familiar	}&	2,650,710,234	&	3,323,679,636	&	3,008,484,214	&	90.5	\\
			\rowcolor{color1!5!white}\multicolumn{1}{l}{	Agua y saneamiento	}&	365,267,394	&	410,584,483	&	112,495,918	&	27.4	\\
			\multicolumn{1}{l}{	Gobernanza local	}&	544,200	&	544,200	&	510,836	&	93.9	\\
			\rowcolor{color1!5!white}\multicolumn{1}{l}{	Escuelas saludables	}&	732,802,593	&	734,161,512	&	629,167,958	&	85.7	\\
			\multicolumn{1}{l}{	Hogar saludable	}&	0	&	178,917,340	&	159,074,023	&	88.9	\\
			\rowcolor{color1!5!white}\multicolumn{1}{l}{	Alfabetización	}&	194,688,339	&	145,382,916	&	107,601,793	&	74.0	\\
			\rowcolor{color1!20!white} \multicolumn{1}{l}{\Bold{	Eje transversal	}}&	45977554	&	65440879	&	53,033,906	&	81.0	\\
			\multicolumn{1}{l}{	Coordinación interinstitucional	}&	12,398,669	&	25,359,495	&	19,708,563	&	77.7	\\
			\rowcolor{color1!5!white}\multicolumn{1}{l}{	Comunicación para la seguridad alimentaria y nutricional	}&	2,165,900	&	3,399,000	&	1,960,205	&	57.7	\\
			\multicolumn{1}{l}{	Participación comunitaria	}&	0	&	31,262	&	25,737	&	82.3	\\
			\rowcolor{color1!5!white}\multicolumn{1}{l}{	Sistema de monitoreo y evaluación	}&	31,412,985	&	36,651,122	&	31,339,402	&	85.5	\\
			[0.05cm]
			\hline
			&&&&\\[-0.36cm]\end{tabular}\addtocounter{Cuadro}{1}
	\end{center}
	{\footnotesize Fuente:  Elaborado por Sesan con datos de Sicoin-Minfin, 2016.}\\[.1cm]}


%%%%%%%%%%%%%%%%%%%%7.1




\begin{landscape}%\fontsize{4mm}{1.9em}\selectfont \setlength{\arrayrulewidth}{01pt}
	%	$\ $\\[-1.8cm]
	%	{\Bold\color{color1!80!black}{Cuadro \theCuadro $\,-$  Mujeres embarazadas al momento de la encuesta, que recibieron atención pre natal, por establecimiento o lugar a donde asistieron; según características varias. }}\\
	%	{\Bold\color{color1!80!black}{República de Guatemala, año 2008/2009. }}\\
	%	\normalsize (Porcentajes)\\[0.4cm]
	
	{\Bold\color{color1!80!black}{Cuadro \theCuadro $\,-$ Ejecución del Plan del Pacto Hambre Cero (PPH0); por mes, según componentes y eje transversal. }}\\
	{\Bold\color{color1!80!black}{República de Guatemala, año 2013.}}\\
	{\color{color1!80!black}{(Quetzales y porcentaje)}}\\[-0.3cm]
	%			\begin{center}\fontsize{3.0mm}{1.3em}\selectfont \setlength{\arrayrulewidth}{0.7pt}
	\begin{center}\fontsize{3mm}{1.5em}
		\selectfont \setlength{\arrayrulewidth}{1pt}
		%		$\ $\\[-2.0cm]
		$\!$\begin{longtable}{x{2.0cm}rrrrrrrrrrrr}
			&&&&\\[-2.5cm] 
			\hline &&&&\\[-0.45cm]  
			\multicolumn{1}{x{3.5cm}}{{\textbf{Componente, eje transversal}}} &	\multicolumn{12}{c}{\textbf{Mes}}\\[0.05cm]\cline{2-13}
			\multicolumn{1}{c}{\textbf{ y grupo institucional }}&\multicolumn{1}{l}{\textbf{Enero}}&\multicolumn{1}{l}{\textbf{Febrero}}&\multicolumn{1}{l}{\textbf{Marzo}}&\multicolumn{1}{l}{\textbf{Abril}}&\multicolumn{1}{l}{\textbf{Mayo}}&\multicolumn{1}{l}{\textbf{Junio}}&\multicolumn{1}{l}{\textbf{Julio}}&\multicolumn{1}{l}{\textbf{Agosto}}&\multicolumn{1}{l}{\textbf{Septiembre}}&\multicolumn{1}{l}{\textbf{Octubre}}&\multicolumn{1}{l}{\textbf{Noviembre}}&\multicolumn{1}{l}{\textbf{Diciembre}}\\%[-.01cm]
			\hline\endhead
			\hline \multicolumn{5}{r}{\textit{Continúa en la siguiente página}} \\
			\endfoot
				\hline \multicolumn{5}{l}{\textbf{Fuente}: Elaborado por Sesan con datos de Sicoin-Minfin, 2016.}\endlastfoot
			\rowcolor{color1!40!white} \multicolumn{1}{p{3.5cm}}{\Bold{	Total presupuesto	}}&	54,616,315	&	130,112,066	&	430,724,684	&	325,802,737	&	671,686,015	&	654,173,041	&	448,632,476	&	1,020,932,077	&	180,229,138	&	339,831,765	&	398,130,050	&	960,841,304	\\
			\rowcolor{color1!20!white} \multicolumn{1}{p{3.5cm}}{\Bold{	Componentes directos	}}&	11,683,854	&	75,988,473	&	132,594,381	&	88,894,411	&	83,287,142	&	219,571,238	&	111,590,673	&	225,220,545	&	105,394,399	&	86,636,049	&	173,916,657	&	230,565,198	\\
			\multicolumn{1}{p{3.5cm}}{	Provisión de servicios básicos de salud y nutrición	}&	3,440,258	&	41,909,945	&	55,047,845	&	36,451,003	&	34,808,538	&	162,997,252	&	51,403,705	&	107,006,824	&	55,532,142	&	40,259,307	&	86,394,594	&	61,312,599	\\
			\rowcolor{color1!5!white}\multicolumn{1}{p{3.5cm}}{	Promoción de lactancia materna y alimentación complementaria	}&	1,239,084	&	291,631	&	9,858,805	&	7,812,985	&	6,427,600	&	9,674,710	&	10,655,321	&	11,178,842	&	8,884,598	&	4,261,486	&	9,529,662	&	9,895,867	\\
			\multicolumn{1}{p{3.5cm}}{	Alimentos fortificados	}&	0	&	0	&	10,000	&	0	&	10,065,682	&	1,704,410	&	11,539,375	&	9,137,907	&	23,017	&	12,631,817	&	855,278	&	12,079,310	\\
			\rowcolor{color1!5!white}\multicolumn{1}{p{3.5cm}}{	Atención a población vulnerable a la inseguridad alimentaria	}&	7,004,512	&	33,786,898	&	67,677,732	&	44,630,423	&	31,985,321	&	45,194,866	&	37,992,273	&	97,896,972	&	40,954,642	&	29,483,440	&	77,137,123	&	147,277,422	\\
			\rowcolor{color1!20!white} \multicolumn{1}{p{3.5cm}}{\Bold{	Componente de viabilidad	}}&	39,725,031	&	51,050,677	&	294,691,653	&	233,515,519	&	584,904,826	&	430,646,690	&	329,301,433	&	788,450,750	&	71,210,934	&	249,705,000	&	220,674,722	&	723,457,506	\\
			\multicolumn{1}{p{3.5cm}}{	Mejoramiento de los ingresos y la economía familiar	}&	4,146,574	&	27,525,533	&	83,185,166	&	70,625,999	&	485,392,233	&	276,613,158	&	282,765,407	&	635,189,237	&	50,488,643	&	230,519,191	&	179,361,674	&	682,671,398	\\
			\rowcolor{color1!5!white}\multicolumn{1}{p{3.5cm}}{	Agua y saneamiento	}&	1,081,808	&	1,411,793	&	5,542,666	&	7,915,141	&	11,851,358	&	7,205,700	&	11,916,707	&	17,584,470	&	9,493,964	&	8,464,478	&	7,742,226	&	22,285,605	\\
			\multicolumn{1}{p{3.5cm}}{	Gobernanza local	}&	650	&	0	&	70,295	&	56,570	&	0	&	32,668	&	38,110	&	111,779	&	18,024	&	22,525	&	68,007	&	92,208	\\
			\rowcolor{color1!5!white}\multicolumn{1}{p{3.5cm}}{	Escuelas saludables	}&	0	&	15,400,665	&	162,830,031	&	125,498,367	&	47,861,592	&	109,059,820	&	16,266,118	&	117,813,212	&	2,127,154	&	5,215,473	&	23,430,344	&	3,665,182	\\
			\multicolumn{1}{p{3.5cm}}{	Hogar saludable	}&	34,496,000	&	0	&	34,281,096	&	23,136,500	&	27,566,175	&	25,345,100	&	4,983,252	&	5,236,000	&	5,914,500	&	0	&		&	-1,884,600	\\
			\rowcolor{color1!5!white}\multicolumn{1}{p{3.5cm}}{	Alfabetización	}&	0	&	6,712,685	&	8,782,399	&	6,282,942	&	12,233,468	&	12,390,243	&	13,331,839	&	12,516,052	&	3,168,650	&	5,483,332	&	10,072,470	&	16,627,712	\\
			\rowcolor{color1!20!white} \multicolumn{1}{p{3.5cm}}{\Bold{	Eje transversal	}}&	3,207,430	&	3,072,916	&	3,438,650	&	3,392,807	&	3,494,047	&	3,955,113	&	7,740,369	&	7,260,781	&	3,623,805	&	3,490,716	&	3,538,671	&	6,818,600	\\
			\multicolumn{1}{p{3.5cm}}{	Coordinación interinstitucional	}&	903,773	&	888,860	&	1,179,139	&	1,081,071	&	1,203,211	&	1,412,805	&	2,894,328	&	2,808,843	&	1,345,139	&	1,138,756	&	1,145,240	&	3,707,397	\\
			\rowcolor{color1!5!white}\multicolumn{1}{p{3.5cm}}{	Comunicación para la seguridad alimentaria y nutricional	}&	147,543	&	132,877	&	113,173	&	107,796	&	118,931	&	117,641	&	195,551	&	154,062	&	195,999	&	209,199	&	152,763	&	314,671	\\
			\multicolumn{1}{p{3.5cm}}{	Participación comunitaria	}&	0	&	0	&	0	&	0	&	0	&	0	&	0	&	1,737	&	0	&	3,250	&	0	&	20,750	\\
			\rowcolor{color1!5!white}\multicolumn{1}{p{3.5cm}}{	Sistema de monitoreo y evaluación	}&	2,156,113	&	2,051,178	&	2,146,338	&	2,203,939	&	2,171,905	&	2,424,666	&	4,650,491	&	4,296,140	&	2,082,668	&	2,139,512	&	2,240,668	&	2,775,783	\\[-0.1cm]
		\end{longtable}\addtocounter{Cuadro}{1}
	\end{center}
\end{landscape}






%%%%%%%%%%%%%%%%%%%8

%
%
%\hoja{
%	\normalsize
%	{\Bold\color{color1!80!black}{Cuadro \theCuadro $\,-$ Ejecución del plan del pacto hambre cero (PPH0);  por criterios de seguimiento, mes y porcentaje de ejecución, según institución (ministerios, secretarías, descentralizadas) y grupo institucional. }}\\
%	{\Bold\color{color1!80!black}{República de Guatemala, año 2014.}}\\
%	{\color{color1!80!black}{(Quetzales y porcentaje)}}\\[-0.3cm]
%	\begin{center}\fontsize{3.0mm}{1.3em}\selectfont \setlength{\arrayrulewidth}{0.7pt}
%		\begin{tabular}{x{4.0cm}rrrc}
%			\hline &&&&\\[-0.4cm]  
%			\multicolumn{1}{x{2.0cm}}{\raisebox{-.3cm}{\textbf{Componente, eje transversal y grupo institucional }}} &	\multicolumn{3}{x{6.3cm}}{\textbf{Criterios de seguimiento (en quetzales)}}&\multicolumn{1}{x{2cm}}{\multirow{2}{*}[-.5mm]{\textbf{Ejecución}}}\\[0.05cm]\cline{2-4}
%			\multicolumn{1}{l}{ }&\multicolumn{4}{l}{ }\\[-.35cm]
%			\multicolumn{1}{x{2.0cm}}{ } &	\multicolumn{1}{x{2cm}}{\textbf{Asignado}}&\multicolumn{1}{x{2cm}}{\textbf{Vigente}}&\multicolumn{1}{x{2cm}}{\textbf{Ejecutado}}&\multicolumn{1}{x{2cm}}{\textbf{(Porcentaje)}}\\[0.05cm]\hline
%			\rowcolor{color1!40!white} \multicolumn{1}{l}{\Bold{	Total presupuesto	}}&	5,249,682,137	&	6,587,699,083	&	5,615,711,668	&	85.2	\\
%			\rowcolor{color1!20!white} \multicolumn{1}{l}{\Bold{	Ministerios	}}&	4,507,844,851	&	5,778,816,812	&	5,175,638,909	&	89.6	\\
%			\multicolumn{1}{l}{	Ministerio de Educación	}&	699,552,593	&	700,740,593	&	618,271,098	&	88.2	\\
%			\rowcolor{color1!5!white}\multicolumn{1}{l}{	Ministerio de Salud Pública y Asistencia Social	}&	663,857,018	&	990,638,317	&	841,129,249	&	84.9	\\
%			\multicolumn{1}{l}{	Ministerio de Agricultura, Ganadería y Alimentación	}&	955,572,082	&	877,715,651	&	790,560,249	&	90.1	\\
%			\rowcolor{color1!5!white}\multicolumn{1}{l}{	Ministerio de Comunicaciones, Infraestructura y Vivienda	}&	1,265,437,403	&	2,133,468,319	&	1,874,645,098	&	87.9	\\
%			\multicolumn{1}{l}{	Ministerio de Ambiente y Recursos Naturales	}&	1,304,562	&	1,350,231	&	525,833	&	38.9	\\
%			\rowcolor{color1!5!white}\multicolumn{1}{l}{	Ministerio de Desarrollo Social	}&	922,121,193	&	1,074,903,701	&	1,050,507,382	&	97.7	\\
%			\rowcolor{color1!20!white} \multicolumn{1}{l}{\Bold{	Secretarías	}}&	93,163,866	&	120,720,925	&	100,761,731	&	83.5	\\
%			\multicolumn{1}{l}{	Secretaría de Coordinación Ejecutiva de la Presidencia	}&	544,200	&	544,200	&	510,836	&	93.9	\\
%			\rowcolor{color1!5!white}\multicolumn{1}{l}{	Secretaría de Obras Sociales de la Esposa del Presidente	}&	43,937,153	&	52,030,887	&	44,785,106	&	86.1	\\
%			\multicolumn{1}{l}{	Secretaría de Seguridad Alimentaria y Nutricional	}&	45,977,554	&	65,440,879	&	53,033,906	&	81.0	\\
%			\rowcolor{color1!5!white}\multicolumn{1}{l}{	Secretaría de Bienestar Social de la Presidencia	}&	2,704,959	&	2,704,959	&	2,431,882	&	89.9	\\
%			\rowcolor{color1!20!white} \multicolumn{1}{l}{\Bold{	Descentralizadas	}}&	670,604,721	&	688,161,346	&	339,311,027	&	49.3	\\
%			\multicolumn{1}{l}{	Instituto de Ciencia y Tecnología Agrícola -ICTA-	}&	38,000,000	&	40,430,500	&	35,150,813	&	86.9	\\
%			\rowcolor{color1!5!white}\multicolumn{1}{l}{	Instituto de Fomento Municipal -Infom-	}&	362,182,524	&	405,108,963	&	107,945,244	&	26.6	\\
%			\multicolumn{1}{l}{	Consejo Nacional de Alfabetización -Conalfa-	}&	194,688,339	&	145,382,916	&	107,601,793	&	74.0	\\
%			\rowcolor{color1!5!white}\multicolumn{1}{l}{	Instituto Nacional de Comercialización Agrícola -Indeca-	}&	12,000,000	&	22,300,000	&	14,489,251	&	65.0	\\
%			\multicolumn{1}{l}{	Fondo de Tierras 	}&	63,733,858	&	74,938,967	&	74,123,927	&	98.9	\\
%			\rowcolor{color1!20!white} \multicolumn{1}{l}{\Bold{	Grupo institucional (total)	}}&	5,249,682,137	&	6,587,699,083	&	5,615,711,668	&	85.2	\\
%			\multicolumn{1}{l}{	Gobierno central	}&	4,579,077,416	&	5,899,537,737	&	5,276,400,640	&	89.4	\\
%			\rowcolor{color1!5!white}\multicolumn{1}{l}{	Descentralizadas	}&	670,604,721	&	688,161,346	&	339,311,027	&	49.5	\\
%			[0.05cm]
%			\hline
%			&&&&\\[-0.36cm]\end{tabular}\addtocounter{Cuadro}{1}
%	\end{center}
%	{\footnotesize Fuente:  Elaborado por Sesan con datos de Sicoin-Minfin, 2016.}\\[.1cm]
%	%	{\parbox{13cm}{\footnotesize \textbf{Notas:} Se denomina desnutrición severa cuando los niños están 3 desviaciones estándar o más por debajo de la media, de acuerdo a la tabla de medidas de la OMS. }}\\[.3cm]
%	%	{\parbox{13cm}{\footnotesize Se denomina desnutrición total cuando los niños están dos desviaciones estándar o más por debajo de la media. Incluye a los niños que están 3 desviaciones estándar o más por debajo de la media.}}\\
%}




%%%%%%%%%%%%%%%%%%%%%8.1




%
%\begin{landscape}%\fontsize{4mm}{1.9em}\selectfont \setlength{\arrayrulewidth}{01pt}
%	%	$\ $\\[-1.8cm]
%	%	{\Bold\color{color1!80!black}{Cuadro \theCuadro $\,-$  Mujeres embarazadas al momento de la encuesta, que recibieron atención pre natal, por establecimiento o lugar a donde asistieron; según características varias. }}\\
%	%	{\Bold\color{color1!80!black}{República de Guatemala, año 2008/2009. }}\\
%	%	\normalsize (Porcentajes)\\[0.4cm]
%	
%	{\Bold\color{color1!80!black}{Cuadro \theCuadro $\,-$ Ejecución del plan del pacto hambre cero (PPH0);  por mes, según institución (ministerios, secretarías, descentralizadas) y grupo institucional. }}\\
%	{\Bold\color{color1!80!black}{República de Guatemala, año 2014.}}\\
%	{\color{color1!80!black}{(Quetzales y porcentaje)}}\\[-0.3cm]
%	%			\begin{center}\fontsize{3.0mm}{1.3em}\selectfont \setlength{\arrayrulewidth}{0.7pt}
%	\begin{center}\fontsize{2.5mm}{1.1em}
%		\selectfont \setlength{\arrayrulewidth}{1pt}
%		%		$\ $\\[-2.0cm]
%		$\!$\begin{longtable}{x{4.5cm}rrrrrrrrrrrr}
%			&&&&\\[-3cm] 
%			\hline &&&&\\[-0.45cm]  
%			\multicolumn{1}{x{4.5cm}}{{\textbf{Componente, eje transversal}}} &	\multicolumn{12}{c}{\textbf{Mes}}\\[0.05cm]\cline{2-13}
%			\multicolumn{1}{c}{\textbf{ y grupo institucional }}&\multicolumn{1}{l}{\textbf{Enero}}&\multicolumn{1}{l}{\textbf{Febrero}}&\multicolumn{1}{l}{\textbf{Marzo}}&\multicolumn{1}{l}{\textbf{Abril}}&\multicolumn{1}{l}{\textbf{Mayo}}&\multicolumn{1}{l}{\textbf{Junio}}&\multicolumn{1}{l}{\textbf{Julio}}&\multicolumn{1}{l}{\textbf{Agosto}}&\multicolumn{1}{l}{\textbf{Septiembre}}&\multicolumn{1}{l}{\textbf{Octubre}}&\multicolumn{1}{l}{\textbf{Noviembre}}&\multicolumn{1}{l}{\textbf{Diciembre}}\\%[-.01cm]
%			\hline\endhead
%			\hline \multicolumn{5}{r}{\textit{Continúa en la siguiente página}} \\
%			\endfoot
%			\rowcolor{color1!40!white} \multicolumn{1}{l}{\Bold{	Total presupuesto	}}&	54,616,315	&	130,112,066	&	430,724,684	&	325,802,737	&	671,686,015	&	654,173,041	&	448,632,476	&	1,020,932,077	&	180,229,138	&	339,831,765	&	398,130,050	&	960,841,304	\\
%			\rowcolor{color1!20!white} \multicolumn{1}{l}{\Bold{	Ministerios	}}&	49,915,812	&	114,745,209	&	399,966,258	&	299,846,069	&	623,249,065	&	623,207,493	&	404,593,566	&	973,170,083	&	154,315,521	&	317,928,950	&	364,997,344	&	849,703,538	\\
%			\multicolumn{1}{p{4.5cm}}{	Ministerio de Educación	}&	0	&	15,400,665	&	162,830,031	&	122,612,017	&	47,861,592	&	108,299,182	&	13,104,148	&	116,621,413	&	2,127,154	&	2,495,519	&	23,254,195	&	3,665,182	\\
%			\rowcolor{color1!5!white}\multicolumn{1}{p{4.5cm}}{	Ministerio de Salud Pública y Asistencia Social	}&	4,694,250	&	42,131,646	&	60,142,163	&	39,362,235	&	46,878,576	&	169,961,384	&	67,104,701	&	121,118,339	&	57,919,578	&	57,286,683	&	92,434,766	&	82,094,929	\\
%			\multicolumn{1}{p{4.5cm}}{	Ministerio de Agricultura, Ganadería y Alimentación	}&	4,153,691	&	7,877,764	&	36,047,278	&	32,708,713	&	62,582,924	&	141,113,324	&	41,268,171	&	167,159,108	&	19,573,670	&	48,408,589	&	30,935,362	&	198,731,656	\\
%			\rowcolor{color1!5!white}\multicolumn{1}{p{4.5cm}}{	Ministerio de Comunicaciones, Infraestructura y Vivienda	}&	34,496,000	&	0	&	34,281,096	&	40,562,850	&	459,942,118	&	158,189,566	&	231,596,109	&	417,990,005	&	36,029,745	&	185,679,440	&	96,139,495	&	179,738,675	\\
%			\multicolumn{1}{p{4.5cm}}{	Ministerio de Ambiente y Recursos Naturales	}&	52,762	&	38,956	&	38,956	&	38,956	&	38,956	&	38,956	&	67,683	&	38,956	&	38,581	&	38,956	&	38,956	&	55,159	\\
%			\rowcolor{color1!5!white}\multicolumn{1}{p{4.5cm}}{	Ministerio de Desarrollo Social	}&	6,519,110	&	49,296,178	&	106,626,734	&	64,561,298	&	5,944,899	&	45,605,081	&	51,452,754	&	150,242,262	&	38,626,793	&	24,019,764	&	122,194,570	&	385,417,939	\\
%			\rowcolor{color1!20!white} \multicolumn{1}{l}{\Bold{	Secretarías	}}&	3,214,562	&	3,162,583	&	9,250,723	&	8,629,457	&	8,152,496	&	8,666,529	&	14,327,745	&	13,921,074	&	10,354,092	&	3,729,548	&	8,366,781	&	8,986,141	\\
%			\multicolumn{1}{p{4.5cm}}{	Secretaría de Coordinación Ejecutiva de la Presidencia	}&	650	&	0	&	70,295	&	56,570	&	0	&	32,668	&	38,110	&	111,779	&	18,024	&	22,525	&	68,007	&	92,208	\\
%			\rowcolor{color1!5!white}\multicolumn{1}{p{4.5cm}}{	Secretaría de Obras Sociales de la Esposa del Presidente	}&	0	&	0	&	5,559,749	&	5,010,036	&	4,373,970	&	4,239,443	&	6,336,026	&	6,296,133	&	6,530,592	&	0	&	4,469,392	&	1,969,765	\\
%			\multicolumn{1}{p{4.5cm}}{	Secretaría de Seguridad Alimentaria y Nutricional	}&	3,207,430	&	3,072,916	&	3,438,650	&	3,392,807	&	3,494,047	&	3,955,113	&	7,740,369	&	7,260,781	&	3,623,805	&	3,490,716	&	3,538,671	&	6,818,600	\\
%			\rowcolor{color1!5!white}\multicolumn{1}{p{4.5cm}}{	Secretaría de Bienestar Social de la Presidencia	}&	6,481	&	89,667	&	182,029	&	170,044	&	284,479	&	439,305	&	213,240	&	252,381	&	181,671	&	216,307	&	290,711	&	105,566	\\
%			\rowcolor{color1!20!white} \multicolumn{1}{l}{\Bold{	Descentralizadas	}}&	1,485,942	&	12,204,273	&	21,507,703	&	17,327,211	&	40,284,454	&	22,299,019	&	29,711,164	&	33,840,919	&	15,559,525	&	18,173,266	&	24,765,925	&	102,151,625	\\
%			\multicolumn{1}{p{4.5cm}}{	Instituto de Ciencia y Tecnología Agrícola -ICTA-	}&		&	3,467,714	&	2,001,345	&	2,534,196	&	3,408,171	&	2,116,657	&	3,312,351	&	2,594,584	&	2,244,874	&	2,953,323	&	2,884,161	&	7,633,436	\\
%			\rowcolor{color1!5!white}\multicolumn{1}{p{4.5cm}}{	Instituto de Fomento Municipal -Infom-	}&	1,007,656	&	1,353,099	&	4,536,419	&	7,597,857	&	11,577,199	&	6,902,985	&	11,793,458	&	17,202,234	&	9,263,298	&	8,075,141	&	7,287,936	&	21,347,962	\\
%			\multicolumn{1}{p{4.5cm}}{	Consejo Nacional de Alfabetización -Conalfa-	}&	0	&	6,712,685	&	8,782,399	&	6,282,942	&	12,233,468	&	12,390,243	&	13,331,839	&	12,516,052	&	3,168,650	&	5,483,332	&	10,072,470	&	16,627,712	\\
%			\rowcolor{color1!5!white}\multicolumn{1}{p{4.5cm}}{	Instituto Nacional de Comercialización Agrícola -Indeca-	}&	478,285	&	542,452	&	640,798	&	785,982	&	830,547	&	760,291	&	1,079,616	&	1,393,829	&	759,220	&	1,535,874	&	866,091	&	4,816,266	\\
%			\multicolumn{1}{p{4.5cm}}{	Fondo de Tierras 	}&	0	&	128,322	&	5,546,741	&	126,234	&	12,235,070	&	128,843	&	193,900	&	134,220	&	123,483	&	125,596	&	3,655,268	&	51,726,251	\\
%			\rowcolor{color1!20!white} \multicolumn{1}{l}{\Bold{	Grupo institucional (total)	}}&	54,616,315	&	130,112,066	&	430,724,684	&	325,802,737	&	671,686,015	&	654,173,041	&	448,632,476	&	1,020,932,077	&	180,229,138	&	339,831,765	&	398,130,050	&	960,841,304	\\
%			\multicolumn{1}{p{4.5cm}}{	Gobierno central	}&	53,130,374	&	117,907,792	&	409,216,982	&	308,475,526	&	631,401,560	&	631,874,021	&	418,921,311	&	987,091,158	&	164,669,613	&	321,658,498	&	373,364,125	&	858,689,679	\\
%			\rowcolor{color1!5!white}\multicolumn{1}{p{4.5cm}}{	Descentralizadas	}&	1,485,942	&	12,204,273	&	21,507,703	&	17,327,211	&	40,284,454	&	22,299,019	&	29,711,164	&	33,840,919	&	15,559,525	&	18,173,266	&	24,765,925	&	102,151,625	\\
%			[-0.28cm]
%		\end{longtable}\addtocounter{Cuadro}{1}
%	\end{center}
%\end{landscape}
%
%
%


%%%%%%%%%%%%%%%%%%%09



\hoja{
	\normalsize
	{\Bold\color{color1!80!black}{Cuadro \theCuadro $\,-$Ejecución del plan del pacto hambre cero (PPH0); por criterios de seguimiento y porcentaje de ejecución, según componentes y eje transversal. }}\\
	%	{\Bold\color{color1!80!black}{ por criterios de seguimiento, mes y porcentaje de ejecución, según institución (ministerios, secretarías, descentralizadas) y grupo institucional. }}\\
	{\Bold\color{color1!80!black}{República de Guatemala, año 2015.}}\\
	{\color{color1!80!black}{(Quetzales y porcentaje)}}\\[-0.3cm]
	\begin{center}\fontsize{3.0mm}{1.3em}\selectfont \setlength{\arrayrulewidth}{0.7pt}
		\begin{tabular}{x{4.0cm}rrrc}
			\hline &&&&\\[-0.4cm]  
			\multicolumn{1}{x{2.0cm}}{\raisebox{-.3cm}{\textbf{Componente, eje transversal y grupo institucional }}} &	\multicolumn{3}{x{6.3cm}}{\textbf{Criterios de seguimiento (en quetzales)}}&\multicolumn{1}{x{2cm}}{\multirow{2}{*}[-.5mm]{\textbf{Ejecución}}}\\[0.05cm]\cline{2-4}
			\multicolumn{1}{l}{ }&\multicolumn{4}{l}{ }\\[-.35cm]
			\multicolumn{1}{x{2.0cm}}{ } &	\multicolumn{1}{x{2cm}}{\textbf{Asignado}}&\multicolumn{1}{x{2cm}}{\textbf{Vigente}}&\multicolumn{1}{x{2cm}}{\textbf{Ejecutado}}&\multicolumn{1}{x{2cm}}{\textbf{(Porcentaje)}}\\[0.05cm]\hline
			\rowcolor{color1!40!white} \multicolumn{1}{l}{\Bold{	Total presupuesto	}}&	5,433,883,259	&	5,342,538,764	&	3,560,292,421	&	66.6	\\
			\rowcolor{color1!20!white} \multicolumn{1}{l}{\Bold{	Componentes directos	}}&	2,107,359,915	&	2,134,720,536	&	1,677,489,889	&	78.6	\\
			\multicolumn{1}{l}{	Provisión de servicios básicos de salud y nutrición	}&	1,097,107,441	&	1,404,609,220	&	1,133,929,226	&	80.7	\\
			\rowcolor{color1!5!white}\multicolumn{1}{l}{	Promoción de lactancia materna y alimentación complementaria	}&	43,944,951	&	109,715,710	&	92,065,076	&	83.9	\\
			\multicolumn{1}{l}{	Alimentos fortificados	}&	71,420,195	&	103,562,337	&	55,948,828	&	54.0	\\
			\rowcolor{color1!5!white}\multicolumn{1}{l}{	Atención a población vulnerable a la inseguridad alimentaria	}&	894,887,328	&	516,833,270	&	395,546,758	&	76.5	\\
			\rowcolor{color1!20!white} \multicolumn{1}{l}{\Bold{	Componente de viabilidad	}}&	3,241,332,009	&	3,153,765,975	&	1,831,822,000	&	58.1	\\
			\multicolumn{1}{l}{	Mejoramiento de los ingresos y la economía familiar	}&	1,689,819,145	&	1,742,949,859	&	863,638,954	&	49.6	\\
			\rowcolor{color1!5!white}\multicolumn{1}{l}{	Agua y saneamiento	}&	229,520,584	&	284,243,748	&	61,745,335	&	21.7	\\
			\multicolumn{1}{l}{	Gobernanza local	}&	3,068,045	&	2,863,616	&	2,540,717	&	88.7	\\
			\rowcolor{color1!5!white}\multicolumn{1}{l}{	Escuelas saludables	}&	752,174,615	&	578,239,444	&	574,875,121	&	99.4	\\
			\multicolumn{1}{l}{	Hogar saludable	}&	391,776,123	&	383,064,990	&	203,436,214	&	53.1	\\
			\rowcolor{color1!5!white}\multicolumn{1}{l}{	Alfabetización	}&	174,973,497	&	162,404,318	&	125,585,659	&	77.3	\\
			\rowcolor{color1!20!white} \multicolumn{1}{l}{\Bold{	Eje transversal	}}&	 85,191,335 	 & 	 54,052,253 	 & 	 50,980,532 	 & 	94.3	 \\ 
			\multicolumn{1}{l}{	Coordinación interinstitucional	}&	48,607,780	&	24,237,708	&	23,613,395	&	97.4	\\
			\rowcolor{color1!5!white}\multicolumn{1}{l}{	Comunicación para la seguridad alimentaria y nutricional	}&	2,992,056	&	1,286,695	&	1,221,818	&	95.0	\\
			\multicolumn{1}{l}{	Participación comunitaria	}&	3,972,884	&	14,070,218	&	13,043,810	&	92.7	\\
			\rowcolor{color1!5!white}\multicolumn{1}{l}{	Sistema de información de seguridad alimentaria y nutricional	}&	2,054,179	&	2,160,938	&	924,523	&	42.8	\\
			\multicolumn{1}{l}{	Sistema de monitoreo y evaluación	}&	27,564,436	&	12,296,694	&	12,176,988	&	99.0	\\
			[0.05cm]
			\hline
			&&&&\\[-0.36cm]\end{tabular}\addtocounter{Cuadro}{1}
	\end{center}
	{\footnotesize Fuente:  Elaborado por Sesan con datos de Sicoin-Minfin, 2016.}\\[.1cm]
	%	{\parbox{13cm}{\footnotesize \textbf{Notas:} Se denomina desnutrición severa cuando los niños están 3 desviaciones estándar o más por debajo de la media, de acuerdo a la tabla de medidas de la OMS. }}\\[.3cm]
	%	{\parbox{13cm}{\footnotesize Se denomina desnutrición total cuando los niños están dos desviaciones estándar o más por debajo de la media. Incluye a los niños que están 3 desviaciones estándar o más por debajo de la media.}}\\
}




%%%%%%%%%%%%%%%%%%%10


%
%\hoja{
%	\normalsize
%	{\Bold\color{color1!80!black}{Cuadro \theCuadro $\,-$ Ejecución del plan del pacto hambre cero (PPH0); por criterios de seguimiento, mes y porcentaje de ejecución, }}\\
%	{\Bold\color{color1!80!black}{según institución (ministerios, secretarías, descentralizadas) y grupo institucional. }}\\
%	{\Bold\color{color1!80!black}{República de Guatemala, año 2015.}}\\
%	{\color{color1!80!black}{(Quetzales y porcentaje)}}\\[-0.3cm]
%	\begin{center}\fontsize{3.0mm}{1.3em}\selectfont \setlength{\arrayrulewidth}{0.7pt}
%		\begin{tabular}{x{4.0cm}rrrc}
%			\hline &&&&\\[-0.4cm]  
%			\multicolumn{1}{x{2.0cm}}{\raisebox{-.3cm}{\textbf{Componente, eje transversal y grupo institucional }}} &	\multicolumn{3}{x{6.3cm}}{\textbf{Criterios de seguimiento (en quetzales)}}&\multicolumn{1}{x{2cm}}{\multirow{2}{*}[-.5mm]{\textbf{Ejecución}}}\\[0.05cm]\cline{2-4}
%			\multicolumn{1}{l}{ }&\multicolumn{4}{l}{ }\\[-.35cm]
%			\multicolumn{1}{x{2.0cm}}{ } &	\multicolumn{1}{x{2cm}}{\textbf{Asignado}}&\multicolumn{1}{x{2cm}}{\textbf{Vigente}}&\multicolumn{1}{x{2cm}}{\textbf{Ejecutado}}&\multicolumn{1}{x{2cm}}{\textbf{(Porcentaje)}}\\[0.05cm]\hline
%			\rowcolor{color1!40!white} \multicolumn{1}{l}{\Bold{	Total presupuesto	}}&	5,433,883,259	&	5,342,538,764	&	3,560,292,421	&	66.6	\\
%			\rowcolor{color1!20!white} \multicolumn{1}{l}{\Bold{	Ministerios	}}&	4,833,238,682	&	4,687,601,799	&	3,235,010,865	&	69.0	\\
%			\multicolumn{1}{l}{	Ministerio de Educación	}&	733,498,088	&	573,507,563	&	573,099,895	&	99.9	\\
%			\rowcolor{color1!5!white}\multicolumn{1}{l}{	Ministerio de Salud Pública y Asistencia Social	}&	1,225,659,293	&	1,590,091,837	&	1,251,483,963	&	78.7	\\
%			\multicolumn{1}{l}{	Ministerio de Agricultura, Ganadería y Alimentación	}&	735,423,108	&	738,297,821	&	368,724,638	&	49.9	\\
%			\rowcolor{color1!5!white}\multicolumn{1}{l}{	Ministerio de Comunicaciones, Infraestructura y Vivienda	}&	1,315,185,738	&	1,211,547,974	&	613,552,025	&	50.6	\\
%			\multicolumn{1}{l}{	Ministerio de Ambiente y Recursos Naturales	}&	6,064,342	&	6,264,563	&	5,616,550	&	89.7	\\
%			\rowcolor{color1!5!white}\multicolumn{1}{l}{	Ministerio de Desarrollo Social	}&	817,408,113	&	567,892,041	&	422,533,794	&	74.4	\\
%			\rowcolor{color1!20!white} \multicolumn{1}{l}{\Bold{	Secretarías	}}&	90,240,558	&	113,289,469	&	103,343,636	&	91.2	\\
%			\multicolumn{1}{l}{	Secretaría de Coordinación Ejecutiva de la Presidencia	}&	3,068,045	&	2,863,616	&	2,540,717	&	88.7	\\
%			\rowcolor{color1!5!white}\multicolumn{1}{l}{	Secretaría de Obras Sociales de la Esposa del Presidente	}&	0	&	54,448,531	&	47,897,317	&	88.0	\\
%			\multicolumn{1}{l}{	Secretaría de Seguridad Alimentaria y Nutricional	}&	85,191,335	&	54,052,253	&	50,980,532	&	94.3	\\
%			\rowcolor{color1!5!white}\multicolumn{1}{l}{	Secretaría de Bienestar Social de la Presidencia	}&	1,981,178	&	1,925,069	&	1,925,069	&	100.0	\\
%			\rowcolor{color1!20!white} \multicolumn{1}{l}{\Bold{	Descentralizadas	}}&	510,404,019	&	541,647,496	&	221,937,920	&	41.0	\\
%			\multicolumn{1}{l}{	Instituto de Ciencia y Tecnología Agrícola -ICTA-	}&	37,500,000	&	37,500,000	&	32,331,296	&	86.2	\\
%			\rowcolor{color1!5!white}\multicolumn{1}{l}{	Instituto de Fomento Municipal -Infom-	}&	208,288,358	&	249,401,014	&	36,765,567	&	14.7	\\
%			\multicolumn{1}{l}{	Consejo Nacional de Alfabetización -Conalfa-	}&	174,973,497	&	162,404,318	&	125,585,659	&	77.3	\\
%			\rowcolor{color1!5!white}\multicolumn{1}{l}{	Instituto Nacional de Comercialización Agrícola -Indeca-	}&	17,000,000	&	19,700,000	&	11,148,774	&	56.6	\\
%			\multicolumn{1}{l}{	Fondo de Tierras 	}&	72,642,164	&	72,642,164	&	16,106,625	&	22.2	\\
%			\rowcolor{color1!20!white} \multicolumn{1}{l}{\Bold{	Grupo institucional (total)	}}&	5,433,883,259	&	5,342,538,764	&	3,560,292,421	&	66.6	\\
%			\multicolumn{1}{l}{	Gobierno central	}&	4,923,479,240	&	4,800,891,268	&	3,338,354,501	&	69.5	\\
%			\rowcolor{color1!5!white}\multicolumn{1}{l}{	Descentralizadas	}&	510,404,019	&	541,647,496	&	221,937,920	&	41.0	\\
%			[0.05cm]
%			\hline
%			&&&&\\[-0.36cm]\end{tabular}\addtocounter{Cuadro}{1}
%	\end{center}
%	{\footnotesize Fuente:  Elaborado por Sesan con datos de Sicoin-Minfin, 2016.}\\[.1cm]
%	%	{\parbox{13cm}{\footnotesize \textbf{Notas:} Se denomina desnutrición severa cuando los niños están 3 desviaciones estándar o más por debajo de la media, de acuerdo a la tabla de medidas de la OMS. }}\\[.3cm]
%	%	{\parbox{13cm}{\footnotesize Se denomina desnutrición total cuando los niños están dos desviaciones estándar o más por debajo de la media. Incluye a los niños que están 3 desviaciones estándar o más por debajo de la media.}}\\
%}


%%%%%%%%%%%%%11



\hoja{
	{\Bold\Large 7.3	Seguimiento Especial del Gasto de la Ventana de los Mil Días. Años 2013 a 2015}\\
	\normalsize
	{\Bold\color{color1!80!black}{Cuadro \theCuadro $\,-$ Presupuesto de la ventana de los mil días, por criterios de seguimiento y porcentaje de ejecución;  según departamento. }}\\
	{\Bold\color{color1!80!black}{República de Guatemala, año 2013.}}\\
	{\color{color1!80!black}{(Quetzales y porcentaje)}}\\[-0.3cm]
	\begin{center}\fontsize{3.0mm}{1.3em}\selectfont \setlength{\arrayrulewidth}{0.7pt}
		\begin{tabular}{x{4.0cm}rrrc}
			\hline &&&&\\[-0.4cm]  
			\multicolumn{1}{x{2.0cm}}{\raisebox{-.3cm}{\textbf{Departamento}}} &	\multicolumn{3}{x{6.3cm}}{\textbf{Criterios de seguimiento (en quetzales)}}&\multicolumn{1}{x{2cm}}{\multirow{2}{*}[-.5mm]{\textbf{Ejecución}}}\\[0.05cm]\cline{2-4}
			\multicolumn{1}{l}{ }&\multicolumn{4}{l}{ }\\[-.35cm]
			\multicolumn{1}{x{2.0cm}}{ } &	\multicolumn{1}{x{2cm}}{\textbf{Asignado}}&\multicolumn{1}{x{2cm}}{\textbf{Vigente}}&\multicolumn{1}{x{2cm}}{\textbf{Ejecutado}}&\multicolumn{1}{x{2cm}}{\textbf{(Porcentaje)}}\\[0.05cm]\hline
			\rowcolor{color1!20!white} \multicolumn{1}{l}{\Bold{	Total presupuesto	 }}& 	130,246,108	&	213,921,494	&	188,260,411	&	88.0	\\
			\multicolumn{1}{l}{	Guatemala	 }& 	73,631,550	&	66,841,657	&	43,962,366	&	65.8	\\
			\rowcolor{color1!5!white}\multicolumn{1}{l}{	El Progreso 	 }& 	856,331	&	2,457,716	&	2,248,041	&	91.5	\\
			\multicolumn{1}{l}{	Sacatepéquez	}&	1,475,900	&	2,084,354	&	2,214,189	&	106.2	\\
			\rowcolor{color1!5!white}\multicolumn{1}{l}{	Chimaltenango 	}&	0	&	4,766,529	&	4,764,157	&	100.0	\\
			\multicolumn{1}{l}{	Escuintla 	}&	94,152	&	3,391,274	&	2,895,890	&	85.4	\\
			\rowcolor{color1!5!white}\multicolumn{1}{l}{	Santa Rosa 	}&	1,029,512	&	3,603,877	&	3,452,175	&	95.8	\\
			\multicolumn{1}{l}{	Sololá	}&	2,982,273	&	4,650,488	&	4,484,018	&	96.4	\\
			\rowcolor{color1!5!white}\multicolumn{1}{l}{	Totonicapán	}&	3,107,727	&	6,233,668	&	6,187,812	&	99.3	\\
			\multicolumn{1}{l}{	Quetzaltenango 	}&	1,742,000	&	9,543,728	&	9,542,685	&	100.0	\\
			\rowcolor{color1!5!white}\multicolumn{1}{l}{	Suchitepéquez	}&	412,491	&	7,450,304	&	7,345,584	&	98.6	\\
			\multicolumn{1}{l}{	Retalhuleu	}&	387,987	&	3,303,147	&	2,920,526	&	88.4	\\
			\rowcolor{color1!5!white}\multicolumn{1}{l}{	San Marcos	}&	10,368,812	&	9,178,458	&	9,093,055	&	99.1	\\
			\multicolumn{1}{l}{	Huehuetenango	}&	3,405,856	&	25,812,645	&	25,496,204	&	98.8	\\
			\rowcolor{color1!5!white}\multicolumn{1}{l}{	Quiché	}&	10,644,589	&	30,445,304	&	29,887,076	&	98.2	\\
			\multicolumn{1}{l}{	Baja Verapaz 	}&	1,886,944	&	3,040,773	&	3,141,200	&	103.3	\\
			\rowcolor{color1!5!white}\multicolumn{1}{l}{	Alta Verapaz	}&	128,647	&	4,004,900	&	3,630,618	&	90.7	\\
			\multicolumn{1}{l}{	Petén	}&	1,420,128	&	7,651,000	&	7,553,364	&	98.7	\\
			\rowcolor{color1!5!white}\multicolumn{1}{l}{	Izabal 	}&	2,778,596	&	2,367,325	&	2,345,316	&	99.1	\\
			\multicolumn{1}{l}{	Zacapa 	}&	6,461,513	&	492,027	&	386,740	&	78.6	\\
			\rowcolor{color1!5!white}\multicolumn{1}{l}{	Chiquimula 	}&	4,938,411	&	14,137,688	&	14,296,696	&	101.1	\\
			\multicolumn{1}{l}{	Jalapa 	}&	1,297,783	&	1,837,448	&	1,829,803	&	99.6	\\
			\rowcolor{color1!5!white}\multicolumn{1}{l}{	Jutiapa 	}&	1,194,906	&	627,184	&	582,896	&	92.9	\\
			[0.05cm]
			\hline
			&&&&\\[-0.36cm]\end{tabular}\addtocounter{Cuadro}{1}
	\end{center}
	{\footnotesize Fuente:  Elaborado por Sesan con datos de Sicoin-Minfin, 2016.}\\[.1cm]
	%	{\parbox{13cm}{\footnotesize \textbf{Notas:} Se denomina desnutrición severa cuando los niños están 3 desviaciones estándar o más por debajo de la media, de acuerdo a la tabla de medidas de la OMS. }}\\[.3cm]
	%	{\parbox{13cm}{\footnotesize Se denomina desnutrición total cuando los niños están dos desviaciones estándar o más por debajo de la media. Incluye a los niños que están 3 desviaciones estándar o más por debajo de la media.}}\\
}




%%%%%%%%%%%%%%%%%%12



\hoja{
	\normalsize
	{\Bold\color{color1!80!black}{Cuadro \theCuadro $\,-$ Presupuesto de la ventana de los mil días, por criterios de seguimiento y porcentaje de ejecución; según tipo de acción.}}\\
	%	{\Bold\color{color1!80!black}{ según departamento. }}\\
	{\Bold\color{color1!80!black}{República de Guatemala, año 2013.}}\\
	{\color{color1!80!black}{(Quetzales y porcentaje)}}\\[-0.3cm]
	\begin{center}\fontsize{3.0mm}{1.3em}\selectfont \setlength{\arrayrulewidth}{0.7pt}
		\begin{tabular}{x{4.0cm}rrrc}
			\hline &&&&\\[-0.4cm]  
			\multicolumn{1}{x{6.0cm}}{\raisebox{-.3cm}{\textbf{ Acción de la ventana de los mil días }}} &	\multicolumn{3}{x{6.3cm}}{\textbf{Criterios de seguimiento (en quetzales)}}&\multicolumn{1}{x{2cm}}{\multirow{2}{*}[-.5mm]{\textbf{Ejecución}}}\\[0.05cm]\cline{2-4}
			\multicolumn{1}{l}{ }&\multicolumn{4}{l}{ }\\[-.35cm]
			\multicolumn{1}{x{6.0cm}}{ } &	\multicolumn{1}{x{2cm}}{\textbf{Asignado}}&\multicolumn{1}{x{2cm}}{\textbf{Vigente}}&\multicolumn{1}{x{2cm}}{\textbf{Ejecutado}}&\multicolumn{1}{x{2cm}}{\textbf{(Porcentaje)}}\\[0.05cm]\hline
			\rowcolor{color1!40!white} \multicolumn{1}{l}{\Bold{	Total presupuesto	}}&	130,246,108	&	213,921,494	&	188,260,411	&	88.0	\\
			\multicolumn{1}{l}{	Promoción y apoyo de la lactancia materna 	}&	70,667,216	&	79,405,083	&	78,010,257	&	98.2	\\
			\rowcolor{color1!5!white}\multicolumn{1}{l}{	Suplementación de zinc terapéutico en el manejo de la diarrea 	}&	12,227,798	&	14,288,689	&	12,221,569	&	85.5	\\
			\multicolumn{1}{l}{	Desparasitación y vacunación de niños y niñas	}&	947,306	&	15,907,914	&	15,466,429	&	97.2	\\
			\rowcolor{color1!5!white}\multicolumn{1}{l}{	Suplementación de vitamina "A" y micronutrientes a niños y niñas	}&	16,268,643	&	72,906,156	&	53,401,734	&	73.2	\\
			\multicolumn{1}{p{6cm}}{	Suplementación con micronutrientes, hierro y ácido fólico a mujer en edad fértil 	}&	30,135,145	&	31,413,652	&	29,160,423	&	92.8	\\
			[0.05cm]
			\hline
			&&&&\\[-0.36cm]\end{tabular}\addtocounter{Cuadro}{1}
	\end{center}
	{\footnotesize Fuente:  Elaborado por Sesan con datos de Sicoin-Minfin, 2016.}\\[.1cm]}











%%%%%%%%%%%%%%%%%%13



%%%%%%%%%%%%%%%%%%


\hoja{
	\normalsize
	{\Bold\color{color1!80!black}{Cuadro \theCuadro $\,-$ Presupuesto de la ventana de los mil días, por criterios de seguimiento y porcentaje de ejecución; según tipo de acción.}}\\
	%	{\Bold\color{color1!80!black}{ según departamento. }}\\
	{\Bold\color{color1!80!black}{República de Guatemala, año 2014.}}\\
	{\color{color1!80!black}{(Quetzales y porcentaje)}}\\[-0.3cm]
	\begin{center}\fontsize{3.0mm}{1.3em}\selectfont \setlength{\arrayrulewidth}{0.7pt}
		\begin{tabular}{x{4.0cm}rrrc}
			\hline &&&&\\[-0.4cm]  
			\multicolumn{1}{x{6.0cm}}{\raisebox{-.3cm}{\textbf{ Acción de la ventana de los mil días }}} &	\multicolumn{3}{x{6.3cm}}{\textbf{Criterios de seguimiento (en quetzales)}}&\multicolumn{1}{x{2cm}}{\multirow{2}{*}[-.5mm]{\textbf{Ejecución}}}\\[0.05cm]\cline{2-4}
			\multicolumn{1}{l}{ }&\multicolumn{4}{l}{ }\\[-.35cm]
			\multicolumn{1}{x{6.0cm}}{ } &	\multicolumn{1}{x{2cm}}{\textbf{Asignado}}&\multicolumn{1}{x{2cm}}{\textbf{Vigente}}&\multicolumn{1}{x{2cm}}{\textbf{Ejecutado}}&\multicolumn{1}{x{2cm}}{\textbf{(Porcentaje)}}\\[0.05cm]\hline
			\rowcolor{color1!20!white} \multicolumn{1}{m{6.0cm}}{\Bold{	Total presupuesto	}}&	380,172,611	&	619,385,547	&	528,769,202	&	85.4	\\
			\multicolumn{1}{l}{	Promoción y Apoyo de la Lactancia Materna 	}&	72,447,524	&	58,967,472	&	44,759,502	&	75.9	\\
			\rowcolor{color1!5!white}\multicolumn{1}{m{6.0cm}}{	Suplementación de Zinc Terapéutico en el Manejo de la Diarrea 	}&	12,227,798	&	11,729,436	&	10,919,687	&	93.1	\\
			\multicolumn{1}{l}{	Desparasitación y Vacunación de Niños y Niñas	}&	250,917,305	&	413,246,999	&	347,907,510	&	84.2	\\
			\rowcolor{color1!5!white}\multicolumn{1}{m{6.0cm}}{	Suplementación de Vitamina "A " y Micronutrientes a Niños y Niñas	}&	14,444,839	&	50,575,279	&	44,675,947	&	88.3	\\
			\multicolumn{1}{m{6.0cm}}{	Suplementación con Micronutrientes, Hierro y Ácido Fólico a Mujer en Edad Fértil 	}&	30,135,145	&	19,343,869	&	17,617,592	&	91.1	\\
			\rowcolor{color1!5!white}\multicolumn{1}{m{6.0cm}}{	Mejoramiento de la Alimentación Complementaría a partir de los seis meses	}&	0	&	61,265,841	&	59,109,715	&	96.5	\\
			\multicolumn{1}{m{6.0cm}}{	Tratamiento. Oportuno de Desnutrición Aguda Moderada y Severa Utilizando ATLC en Comunidad con Orientación  y Seguimiento  del Personal de Salud	}&	0	&	4,256,651	&	3,779,250	&	88.8	\\
			[0.05cm]
			\hline
			&&&&\\[-0.36cm]\end{tabular}\addtocounter{Cuadro}{1}
	\end{center}
	{\footnotesize Fuente:  Elaborado por Sesan con datos de Sicoin-Minfin, 2016.}\\[.1cm]





%%%%%%%%%%%%%%%%%%14



	\normalsize
	{\Bold\color{color1!80!black}{Cuadro \theCuadro $\,-$Presupuesto de la ventana de los mil días, por criterios de seguimiento y porcentaje de ejecución; según departamento. }}\\
	{\Bold\color{color1!80!black}{República de Guatemala, año 2015.}}\\
	{\color{color1!80!black}{(Quetzales y porcentaje)}}\\[-0.3cm]
	\begin{center}\fontsize{3.0mm}{1.3em}\selectfont \setlength{\arrayrulewidth}{0.7pt}
		\begin{tabular}{x{4.0cm}rrrc}
			\hline &&&&\\[-0.4cm]  
			\multicolumn{1}{x{4.0cm}}{\raisebox{-.3cm}{\textbf{ Departamento}}} &	\multicolumn{3}{x{6.3cm}}{\textbf{Criterios de seguimiento (en quetzales)}}&\multicolumn{1}{x{2cm}}{\multirow{2}{*}[-.5mm]{\textbf{Ejecución}}}\\[0.05cm]\cline{2-4}
			\multicolumn{1}{l}{ }&\multicolumn{4}{l}{ }\\[-.35cm]
			\multicolumn{1}{c}{ } &	\multicolumn{1}{x{2cm}}{\textbf{Asignado}}&\multicolumn{1}{x{2cm}}{\textbf{Vigente}}&\multicolumn{1}{x{2cm}}{\textbf{Ejecutado}}&\multicolumn{1}{x{2cm}}{\textbf{(Porcentaje)}}\\[0.05cm]\hline
			\rowcolor{color1!20!white} \multicolumn{1}{l}{\Bold{	Total presupuesto	}}&	705,840,558	&	1,008,060,707	&	786,358,785	&	78.0	\\
			\multicolumn{1}{l}{	Guatemala	}&	408,306,813	&	537,848,177	&	411,191,010	&	76.5	\\
			\rowcolor{color1!5!white}\multicolumn{1}{l}{	El Progreso 	}&	1,204,080	&	5,621,153	&	3,336,247	&	59.4	\\
			\multicolumn{1}{l}{	Sacatepéquez	}&	5,845,429	&	7,542,807	&	6,566,065	&	87.1	\\
			\rowcolor{color1!5!white}\multicolumn{1}{l}{	Chimaltenango 	}&	13,722,485	&	16,447,663	&	13,390,492	&	81.4	\\
			\multicolumn{1}{l}{	Escuintla 	}&	10,385,375	&	11,548,853	&	9,906,879	&	85.8	\\
			\rowcolor{color1!5!white}\multicolumn{1}{l}{	Santa Rosa 	}&	5,939,642	&	18,638,051	&	12,168,429	&	65.3	\\
			\multicolumn{1}{l}{	Sololá	}&	17,311,633	&	26,007,672	&	25,349,570	&	97.5	\\
			\rowcolor{color1!5!white}\multicolumn{1}{l}{	Totonicapán	}&	12,382,402	&	7,669,164	&	5,631,303	&	73.4	\\
			\multicolumn{1}{l}{	Quetzaltenango 	}&	13,819,115	&	25,290,106	&	21,919,804	&	86.7	\\
			\rowcolor{color1!5!white}\multicolumn{1}{l}{	Suchitepéquez	}&	17,305,413	&	20,870,716	&	14,276,700	&	68.4	\\
			\multicolumn{1}{l}{	Retalhuleu	}&	4,456,939	&	10,180,064	&	6,692,518	&	65.7	\\
			\rowcolor{color1!5!white}\multicolumn{1}{l}{	San Marcos	}&	25,675,011	&	30,959,245	&	26,263,304	&	84.8	\\
			\multicolumn{1}{l}{	Huehuetenango	}&	41,779,295	&	72,814,289	&	59,903,539	&	82.3	\\
			\rowcolor{color1!5!white}\multicolumn{1}{l}{	Quiché	}&	42,194,200	&	62,881,390	&	54,537,007	&	86.7	\\
			\multicolumn{1}{l}{	Baja Verapaz 	}&	8,808,891	&	19,214,922	&	16,288,905	&	84.8	\\
			\rowcolor{color1!5!white}\multicolumn{1}{l}{	Alta Verapaz	}&	33,737,037	&	44,675,523	&	32,516,167	&	72.8	\\
			\multicolumn{1}{l}{	Petén	}&	8,300,739	&	16,855,128	&	13,852,908	&	82.2	\\
			\rowcolor{color1!5!white}\multicolumn{1}{l}{	Izabal 	}&	3,875,689	&	13,092,068	&	6,616,147	&	50.5	\\
			\multicolumn{1}{l}{	Zacapa 	}&	3,175,446	&	7,909,827	&	4,432,697	&	56.0	\\
			\rowcolor{color1!5!white}\multicolumn{1}{l}{	Chiquimula 	}&	20,055,885	&	24,222,130	&	23,224,325	&	95.9	\\
			\multicolumn{1}{l}{	Jalapa 	}&	1,248,744	&	6,947,052	&	3,251,617	&	46.8	\\
			\rowcolor{color1!5!white}\multicolumn{1}{l}{	Jutiapa 	}&	6,310,295	&	20,824,707	&	15,043,153	&	72.2	\\
			[0.05cm]
			\hline
			&&&&\\[-0.36cm]\end{tabular}\addtocounter{Cuadro}{1}
	\end{center}
	{\footnotesize Fuente:  Elaborado por Sesan con datos de Sicoin-Minfin, 2016.}\\[.1cm]}


%%%%%%%%%%%%%%%%%%15


\hoja{
	\normalsize
	{\Bold\color{color1!80!black}{Cuadro \theCuadro $\,-$Presupuesto de la ventana de los mil días, por criterios de seguimiento y porcentaje de ejecución; según departamento. }}\\
	{\Bold\color{color1!80!black}{República de Guatemala, año 2015.}}\\
	{\color{color1!80!black}{(Quetzales y porcentaje)}}\\[-0.3cm]
	\begin{center}\fontsize{3.0mm}{1.3em}\selectfont \setlength{\arrayrulewidth}{0.7pt}
		\begin{tabular}{x{5.5cm}rrrc}
			\hline &&&&\\[-0.4cm]  
			\multicolumn{1}{x{5.5cm}}{\raisebox{-.3cm}{\textbf{Acción de la ventana de los mil días }}} &	\multicolumn{3}{x{6.6cm}}{\textbf{Criterios de seguimiento (en quetzales)}}&\multicolumn{1}{x{2cm}}{\multirow{2}{*}[-.5mm]{\textbf{Ejecución}}}\\[0.05cm]\cline{2-4}
			\multicolumn{1}{l}{ }&\multicolumn{4}{l}{ }\\[-.35cm]
			\multicolumn{1}{c}{ } &	\multicolumn{1}{x{2cm}}{\textbf{Asignado}}&\multicolumn{1}{x{2cm}}{\textbf{Vigente}}&\multicolumn{1}{x{2cm}}{\textbf{Ejecutado}}&\multicolumn{1}{x{2cm}}{\textbf{(Porcentaje)}}\\[0.05cm]\hline
			\rowcolor{color1!20!white} \multicolumn{1}{p{5.5cm}}{\Bold{	Total presupuesto	}}&	705,840,558	&	1,008,060,707	&	786,358,785	&	78.0	\\
			\multicolumn{1}{p{5.5cm}}{	Promoción y apoyo de la lactancia materna 	}&	43,944,951	&	55,267,179	&	44,701,659	&	80.9	\\
			\rowcolor{color1!5!white}\multicolumn{1}{p{5.5cm}}{	Suplementación de zinc terapéutico en el manejo de la diarrea 	}&	13,937,987	&	12,429,349	&	9,003,089	&	72.4	\\
			\multicolumn{1}{p{5.5cm}}{	Suplementación de vitamina "A" y micronutrientes a niños y niñas	}&	250,608,099	&	205,354,606	&	99,898,656	&	48.6	\\
			\rowcolor{color1!5!white}\multicolumn{1}{p{5.5cm}}{	Desparasitación y vacunación de niños y niñas	}&	299,792,527	&	611,228,619	&	561,531,908	&	91.9	\\
			\multicolumn{1}{p{5.5cm}}{	Suplementación con micronutrientes, hierro y ácido fólico a mujer en edad fértil 	}&	26,136,799	&	20,218,617	&	15,274,645	&	75.5	\\
			\rowcolor{color1!5!white}\multicolumn{1}{p{5.5cm}}{	Mejoramiento de la alimentación complementaría a partir de los seis meses	}&	69,719,062	&	102,921,967	&	55,819,161	&	54.2	\\
			\multicolumn{1}{p{5.5cm}}{	Fortificción de los alimentos básicos con micronutrientes	}&	1,701,133	&	640,370	&	129,667	&	20.2	\\
			[0.05cm]
			\hline
			&&&&\\[-0.36cm]\end{tabular}\addtocounter{Cuadro}{1}
	\end{center}
	{\footnotesize Fuente:  Elaborado por Sesan con datos de Sicoin-Minfin, 2016.}\\[.1cm]}



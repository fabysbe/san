\INEchaptercarta{Situación y atención a la desnutrición/malnutrición}{}

\addtocounter{Cuadro}{1}
\hoja{{\Bold\color{color1!80!black}{\Large SITUACIÓN NUTRICIONAL DE LA MADRE}}
	
	\normalsize
	{\Bold\color{color1!80!black}{Cuadro \theCuadro $\,-$  Mujeres en edad fértil y menores de 5 años con anemia; según características varias.}}\\
%	{\Bold\color{color1!80!black}{según características varias. }}\\
	{\Bold\color{color1!80!black}{República de Guatemala, año 2008/2009.}}\\
	{\color{color1!80!black}{(Porcentajes)}}\\
	\begin{center}\fontsize{3mm}{1.4em}\selectfont \setlength{\arrayrulewidth}{0.7pt}
		\begin{tabular}{x{5cm}ccc}
			\hline &&&\\[-0.36cm]  
			\multicolumn{1}{x{2.7cm}}{\textbf{Característica}} &	\multicolumn{2}{c}{\Bold{Mujeres en edad fértil}}&\multicolumn{1}{c}{\multirow{2}{*}[1mm]{\Bold{Menores de 5 años}}}\\[0.05cm]\cline{2-3}
			\multicolumn{1}{x{2.7cm}}{ } &	\multicolumn{1}{c}{\textbf{ No embarazadas}}&\multicolumn{1}{c}{\Bold{Embarazadas}}& \multicolumn{1}{c}{ }\\[0.05cm]
\rowcolor{color1!40!white} \multicolumn{1}{l}{\Bold{	Total República	 }}& 	 21.4 	 & 	 29.1 	 & 	 47.7 	 \\ 
\rowcolor{color1!20!white} \multicolumn{1}{l}{\Bold{	Área geográfica	 }}& 		 & 		 & 		 \\ 
\multicolumn{1}{l}{	Urbana	 }& 	 19.1 	 & 	 27.5 	 & 	 46.2 	 \\ 
\rowcolor{color1!5!white}\multicolumn{1}{l}{	Rural	 }& 	 23.1 	 & 	 30.0 	 & 	 48.6 	 \\ 
\rowcolor{color1!20!white} \multicolumn{1}{l}{\Bold{	Departamento	 }}& 		 & 		 & 		 \\ 
\multicolumn{1}{l}{	Guatemala	 }& 	 16.6 	 & 	 30.1 	 & 	 40.7 	 \\ 
\rowcolor{color1!5!white}\multicolumn{1}{l}{	El Progreso	 }& 	 20.8 	 & 	 19.4 	 & 	 37.8 	 \\ 
\multicolumn{1}{l}{	Sacatepéquez	 }& 	 19.9 	 & 	 24.9 	 & 	 54.2 	 \\ 
\rowcolor{color1!5!white}\multicolumn{1}{l}{	Chimaltenango	 }& 	 20.5 	 & 	 24.7 	 & 	 53.5 	 \\ 
\multicolumn{1}{l}{	Escuintla	 }& 	 22.1 	 & 	 27.4 	 & 	 50.2 	 \\ 
\rowcolor{color1!5!white}\multicolumn{1}{l}{	Santa Rosa	 }& 	 12.9 	 & 	 22.3 	 & 	 51.4 	 \\ 
\multicolumn{1}{l}{	Sololá	 }& 	 22.6 	 & 	 19.8 	 & 	 56.1 	 \\ 
\rowcolor{color1!5!white}\multicolumn{1}{l}{	Totonicapán	 }& 	 32.3 	 & 	 36.3 	 & 	 62.2 	 \\ 
\multicolumn{1}{l}{	Quetzaltenango	 }& 	 20.6 	 & 	 35.3 	 & 	 40.2 	 \\ 
\rowcolor{color1!5!white}\multicolumn{1}{l}{	Suchitepéquez	 }& 	 21.1 	 & 	 31.5 	 & 	 37.7 	 \\ 
\multicolumn{1}{l}{	Retalhuleu	 }& 	 23.7 	 & 	 44.8 	 & 	 45.3 	 \\ 
\rowcolor{color1!5!white}\multicolumn{1}{l}{	San Marcos	 }& 	 29.0 	 & 	 34.0 	 & 	 52.6 	 \\ 
\multicolumn{1}{l}{	Huehuetenango	 }& 	 21.1 	 & 	 17.9 	 & 	 47.7 	 \\ 
\rowcolor{color1!5!white}\multicolumn{1}{l}{	Quiché	 }& 	 24.8 	 & 	 30.3 	 & 	 47.4 	 \\ 
\multicolumn{1}{l}{	Baja Verapaz	 }& 	 19.3 	 & 	 31.9 	 & 	 49.8 	 \\ 
\rowcolor{color1!5!white}\multicolumn{1}{l}{	Alta Verapaz	 }& 	 22.2 	 & 	 33.3 	 & 	 46.1 	 \\ 
\multicolumn{1}{l}{	Petén	 }& 	 21.3 	 & 	 34.1 	 & 	 48.5 	 \\ 
\rowcolor{color1!5!white}\multicolumn{1}{l}{	Izabal	 }& 	 35.3 	 & 	 36.9 	 & 	 53.0 	 \\ 
\multicolumn{1}{l}{	Zacapa	 }& 	 20.8 	 & 	 19.4 	 & 	 53.7 	 \\ 
\rowcolor{color1!5!white}\multicolumn{1}{l}{	Chiquimula	 }& 	 27.3 	 & 	 41.3 	 & 	 55.5 	 \\ 
\multicolumn{1}{l}{	Jalapa	 }& 	 15.0 	 & 	 7.3 	 & 	 43.9 	 \\ 
\rowcolor{color1!5!white}\multicolumn{1}{l}{	Jutiapa	 }& 	 13.3 	 & 	 21.3 	 & 	 50.3 	 \\ 
\rowcolor{color1!20!white} \multicolumn{1}{l}{\Bold{	Categoría étnica	 }}& 		 & 		 & 		 \\ 
\multicolumn{1}{l}{	Indígena	 }& 	 24.9 	 & 	 32.2 	 & 	 49.5 	 \\ 
\rowcolor{color1!5!white}\multicolumn{1}{l}{	Ladino	 }& 	 19.0 	 & 	 26.6 	 & 	 46.3 	 \\ 
\rowcolor{color1!20!white} \multicolumn{1}{l}{\Bold{	Nivel de educación	 }}& 		 & 		 & 		 \\ 
\multicolumn{1}{l}{	Sin educación	 }& 	 27.8 	 & 	 33.0 	 & 	 48.3 	 \\ 
\rowcolor{color1!5!white}\multicolumn{1}{l}{	Primaria	 }& 	 20.8 	&	 28.8 	&	 49.2 	 \\ 
\multicolumn{1}{l}{	Secundaria	 }& 	 16.3 	&	 26.2 	&	 44.0 	 \\ 
\rowcolor{color1!5!white}\multicolumn{1}{l}{	Superior	 }& 	 15.6 	&	 21.4 	&	 36.0 	 \\ 
[0.05cm]
			\hline
			&&&\\[-0.36cm]
			\multicolumn{4}{l}{\footnotesize Fuente:  Encuesta Nacional de Condiciones de Vida (Encovi), 2014.}\\
		\end{tabular}\addtocounter{Cuadro}{1}
 	\end{center}}




%%%%5555555555555555%%%%%%%%%%%%%5


%%%%%%%%%%%%%%%%%%


\hoja{
	\normalsize
	{\Bold\color{color1!80!black}{Cuadro \theCuadro $\,-$  Distribución de las mujeres no embarazadas, por valor del Índice de Masa Corporal; }}\\
	{\Bold\color{color1!80!black}{según características varias. }}\\
	{\Bold\color{color1!80!black}{República de Guatemala, año 2008/2009.}}\\
	{\color{color1!80!black}{(Porcentajes)}}\\[-1cm]
	\begin{center}\fontsize{3.5mm}{1.4em}\selectfont \setlength{\arrayrulewidth}{0.7pt}
		\begin{tabular}{x{5cm}ccccc}
			\hline &&&&\\[-0.5cm]  
			\multicolumn{1}{x{2.7cm}}{\raisebox{-.5cm}{\textbf{Característica}}} &	\multicolumn{4}{c}{\raisebox{-.15cm}{\Bold{Índice de masa corporal (IMC)}}}\\[0.05cm]\cline{2-5}
			\multicolumn{1}{x{2.7cm}}{ } &	\multicolumn{1}{x{2.7cm}}{\textbf{Bajo}}&\multicolumn{1}{x{2.7cm}}{\Bold{Normal}}& \multicolumn{1}{x{2.7cm}}{\textbf{Sobre Peso}} &\multicolumn{1}{c}{\textbf{Obesidad}}\\[0.05cm]
			\rowcolor{color1!40!white} \multicolumn{1}{l}{\Bold{	Total	 }}& 	1.6	 & 	47.9	 & 	35.1	 & 	15.4	 \\ 
			\rowcolor{color1!20!white} \multicolumn{1}{l}{\Bold{	Área geográfica	 }}& 		 & 		 & 		 & 		 \\ 
			\multicolumn{1}{l}{	Urbana	 }& 	 1.8 	 & 	 40.5 	 & 	 37.5 	 & 	 20.3 	 \\ 
			\rowcolor{color1!5!white}\multicolumn{1}{l}{	Rural	 }& 	 1.5 	 & 	 53.0 	 & 	 33.4 	 & 	 12.1 	 \\ 
			\rowcolor{color1!20!white} \multicolumn{1}{l}{\Bold{	Departamentos	 }}& 		 & 		 & 		 & 		 \\ 
			\multicolumn{1}{l}{	Guatemala	 }& 	 2.5 	 & 	 38.7 	 & 	 37.3 	 & 	 21.5 	 \\ 
			\rowcolor{color1!5!white}\multicolumn{1}{l}{	El Progreso	 }& 	 1.2 	 & 	 48.7 	 & 	 31.9 	 & 	 18.3 	 \\ 
			\multicolumn{1}{l}{	Sacatepéquez	 }& 	 -   	 & 	 39.2 	 & 	 38.9 	 & 	 21.9 	 \\ 
			\rowcolor{color1!5!white}\multicolumn{1}{l}{	Chimaltenango	 }& 	 1.0 	 & 	 41.1 	 & 	 43.7 	 & 	 14.2 	 \\ 
			\multicolumn{1}{l}{	Escuintla	 }& 	 4.5 	 & 	 38.6 	 & 	 32.4 	 & 	 24.5 	 \\ 
			\rowcolor{color1!5!white}\multicolumn{1}{l}{	Santa Rosa	 }& 	 2.0 	 & 	 44.7 	 & 	 33.2 	 & 	 20.1 	 \\ 
			\multicolumn{1}{l}{	Sololá	 }& 	 1.6 	 & 	 53.3 	 & 	 32.5 	 & 	 12.6 	 \\ 
			\rowcolor{color1!5!white}\multicolumn{1}{l}{	Totonicapán	 }& 	 0.6 	 & 	 51.8 	 & 	 36.5 	 & 	 11.1 	 \\ 
			\multicolumn{1}{l}{	Quetzaltenango	 }& 	 0.8 	 & 	 45.3 	 & 	 39.6 	 & 	 14.3 	 \\ 
			\rowcolor{color1!5!white}\multicolumn{1}{l}{	Suchitepéquez	 }& 	 2.0 	 & 	 39.8 	 & 	 39.0 	 & 	 19.2 	 \\ 
			\multicolumn{1}{l}{	Retalhuleu	 }& 	 1.5 	 & 	 42.5 	 & 	 35.2 	 & 	 20.8 	 \\ 
			\rowcolor{color1!5!white}\multicolumn{1}{l}{	San Marcos	 }& 	 0.8 	 & 	 53.9 	 & 	 34.7 	 & 	 10.7 	 \\ 
			\multicolumn{1}{l}{	Huehuetenango	 }& 	 0.7 	 & 	 62.2 	 & 	 31.3 	 & 	 5.8 	 \\ 
			\rowcolor{color1!5!white}\multicolumn{1}{l}{	Quiché	 }& 	 1.0 	 & 	 54.3 	 & 	 36.7 	 & 	 8.0 	 \\ 
			\multicolumn{1}{l}{	Baja Verapaz	 }& 	 1.6 	 & 	 60.4 	 & 	 28.8 	 & 	 9.1 	 \\ 
			\rowcolor{color1!5!white}\multicolumn{1}{l}{	Alta Verapaz	 }& 	 0.4 	 & 	 51.6 	 & 	 36.3 	 & 	 11.7 	 \\ 
			\multicolumn{1}{l}{	Petén	 }& 	 1.6 	 & 	 43.4 	 & 	 30.6 	 & 	 24.4 	 \\ 
			\rowcolor{color1!5!white}\multicolumn{1}{l}{	Izabal	 }& 	 1.3 	 & 	 42.5 	 & 	 39.0 	 & 	 17.2 	 \\ 
			\multicolumn{1}{l}{	Zacapa	 }& 	 2.0 	 & 	 49.5 	 & 	 31.1 	 & 	 17.4 	 \\ 
			\rowcolor{color1!5!white}\multicolumn{1}{l}{	Chiquimula	 }& 	 0.7 	 & 	 61.7 	 & 	 27.3 	 & 	 10.4 	 \\ 
			\multicolumn{1}{l}{	Jalapa	 }& 	 2.0 	 & 	 56.5 	 & 	 28.8 	 & 	 12.7 	 \\ 
			\rowcolor{color1!5!white}\multicolumn{1}{l}{	Jutiapa	 }& 	 2.4 	 & 	 50.3 	 & 	 32.1 	 & 	 15.2 	 \\ 
			\rowcolor{color1!20!white} \multicolumn{1}{l}{\Bold{	Categoría étnica	 }}& 		 & 		 & 		 & 		 \\ 
			\multicolumn{1}{l}{	Indígena	 }& 	 0.7 	 & 	 52.5 	 & 	 35.3 	 & 	 11.5 	 \\ 
			\rowcolor{color1!5!white}\multicolumn{1}{l}{	Ladino	 }& 	 2.2 	 & 	 44.6 	 & 	 34.9 	 & 	 18.3 	 \\ 
			\rowcolor{color1!20!white} \multicolumn{1}{l}{\Bold{	Nivel de educación	 }}& 		 & 		 & 		 & 		 \\ 
			\multicolumn{1}{l}{	Sin educación	 }& 	 0.8 	 & 	 53.5 	 & 	 33.8 	 & 	 12.0 	 \\ 
			\rowcolor{color1!5!white}\multicolumn{1}{l}{	Primaria	 }& 	 1.6 	 & 	 47.2 	 & 	 35.2 	 & 	 16.0 	 \\ 
			\multicolumn{1}{l}{	Secundaria	 }& 	 2.5 	 & 	 44.3 	 & 	 35.6 	 & 	 17.6 	 \\ 
			\rowcolor{color1!5!white}\multicolumn{1}{l}{	Superior	 }& 	 1.5 	 & 	 38.7 	 & 	 39.8 	 & 	 19.9 	 \\ 
			[0.05cm]
			\hline
			&&&&\\[-0.36cm]
			\multicolumn{5}{l}{\footnotesize Fuente:  INE. Encuesta Nacional de Salud Materno Infantil (Ensmi) 2008/2009.}\\[-.09cm]
			\multicolumn{5}{l}{\parbox{13cm}{\footnotesize \textbf{Nota:} Peso bajo es el menor a 18.4 onzas; peso normal es entre 18.5 y 24.9; el sobre peso al nacer es el mayor a 25.0 y menor 29.9 onzas; obesidad es mayor a 30.0 onzas.}}
		\end{tabular}\addtocounter{Cuadro}{1}
	\end{center}}
	


%%%%%%%%%%%%%%%%%%%



%%%%%%%%%%%%%%%%%%


\hoja{
	\normalsize
	{\Bold\color{color1!80!black}{Cuadro \theCuadro $\,-$ Indicadores de talla de las madres de niños y niñas menores de 5 años; según características varias. }}\\
	%	{\Bold\color{color1!80!black}{según características varias. }}\\
	{\Bold\color{color1!80!black}{República de Guatemala, año 2008/2009.}}\\
	{\color{color1!80!black}{(Centímetros)}}\\%[-1cm]
	\begin{center}\fontsize{3.5mm}{1.4em}\selectfont \setlength{\arrayrulewidth}{0.7pt}
		\begin{tabular}{x{4cm}cc}
			\hline &&\\[-0.5cm]  
			\multicolumn{1}{x{2.7cm}}{\textbf{Característica}} &	\multicolumn{1}{x{3cm}}{\textbf{Talla promedio}}&\multicolumn{1}{x{4cm}}{\Bold{Mide menos de 145cms}}\\[0.05cm]
			\multicolumn{1}{x{2.7cm}}{} &	\multicolumn{1}{x{2.7cm}}{(en centímetros)}&\multicolumn{1}{x{3cm}}{(en porcentaje)}\\[0.05cm]
			\rowcolor{color1!40!white} \multicolumn{1}{l}{\Bold{	Total 	 }}& 	 148.0 	 & 	 31.2 	 \\ 
			\rowcolor{color1!20!white} \multicolumn{1}{l}{\Bold{	Área geográfica	 }}& 		 & 		 \\ 
			\multicolumn{1}{l}{	     Urbana	 }& 	 149.4 	 & 	 25.0 	 \\ 
			\rowcolor{color1!5!white}\multicolumn{1}{l}{	     Rural	 }& 	 147.1 	 & 	 35.4 	 \\ 
			\rowcolor{color1!20!white} \multicolumn{1}{l}{\Bold{	Departamentos	 }}& 		 & 		 \\ 
			\multicolumn{1}{l}{	     Guatemala	 }& 	 149.9 	 & 	 20.5 	 \\ 
			\rowcolor{color1!5!white}\multicolumn{1}{l}{	     El Progreso	 }& 	 151.8 	 & 	 9.9 	 \\ 
			\multicolumn{1}{l}{	     Sacatepéquez	 }& 	 147.5 	 & 	 32.8 	 \\ 
			\rowcolor{color1!5!white}\multicolumn{1}{l}{	     Chimaltenango	 }& 	 146.9 	 & 	 42.1 	 \\ 
			\multicolumn{1}{l}{	     Escuintla	 }& 	 150.6 	 & 	 13.0 	 \\ 
			\rowcolor{color1!5!white}\multicolumn{1}{l}{	     Santa Rosa	 }& 	 150.8 	 & 	 14.7 	 \\ 
			\multicolumn{1}{l}{	     Sololá	 }& 	 144.4 	 & 	 55.7 	 \\ 
			\rowcolor{color1!5!white}\multicolumn{1}{l}{	     Totonicapán	 }& 	 145.3 	 & 	 50.4 	 \\ 
			\multicolumn{1}{l}{	     Quetzaltenango	 }& 	 147.7 	 & 	 34.0 	 \\ 
			\rowcolor{color1!5!white}\multicolumn{1}{l}{	     Suchitepéquez	 }& 	 148.2 	 & 	 31.4 	 \\ 
			\multicolumn{1}{l}{	     Retalhuleu	 }& 	 150.1 	 & 	 18.2 	 \\ 
			\rowcolor{color1!5!white}\multicolumn{1}{l}{	     San Marcos	 }& 	 146.9 	 & 	 33.5 	 \\ 
			\multicolumn{1}{l}{	     Huehuetenango	 }& 	 145.2 	 & 	 47.3 	 \\ 
			\rowcolor{color1!5!white}\multicolumn{1}{l}{	     Quiché	 }& 	 144.5 	 & 	 53.9 	 \\ 
			\multicolumn{1}{l}{	     Baja Verapaz	 }& 	 148.0 	 & 	 32.9 	 \\ 
			\rowcolor{color1!5!white}\multicolumn{1}{l}{	     Alta Verapaz	 }& 	 146.4 	 & 	 40.8 	 \\ 
			\multicolumn{1}{l}{	     Petén	 }& 	 149.0 	 & 	 22.4 	 \\ 
			\rowcolor{color1!5!white}\multicolumn{1}{l}{	     Izabal	 }& 	 149.8 	 & 	 21.1 	 \\ 
			\multicolumn{1}{l}{	     Zacapa	 }& 	 150.5 	 & 	 18.3 	 \\ 
			\rowcolor{color1!5!white}\multicolumn{1}{l}{	     Chiquimula	 }& 	 147.5 	 & 	 34.4 	 \\ 
			\multicolumn{1}{l}{	     Jalapa	 }& 	 149.2 	 & 	 20.8 	 \\ 
			\rowcolor{color1!5!white}\multicolumn{1}{l}{	     Jutiapa	 }& 	 151.4 	 & 	 17.3 	 \\ 
			\rowcolor{color1!20!white} \multicolumn{1}{l}{\Bold{	Categoría étnica	 }}& 		 & 		 \\ 
			\multicolumn{1}{l}{	     Indígena	 }& 	 145.3 	 & 	 48.3 	 \\ 
			\rowcolor{color1!5!white}\multicolumn{1}{l}{	     Ladino	 }& 	 150.0 	 & 	 19.0 	 \\ 
			\rowcolor{color1!20!white} \multicolumn{1}{l}{\Bold{	Nivel de educación	 }}& 		 & 		 \\ 
			\multicolumn{1}{l}{	     Sin educación	 }& 	 145.3 	 & 	 48.8 	 \\ 
			\rowcolor{color1!5!white}\multicolumn{1}{l}{	     Primaria	 }& 	 147.8 	 & 	 30.6 	 \\ 
			\multicolumn{1}{l}{	     Secundaria o más	 }& 	 151.6 	 & 	 13.8 	 \\[0.05cm]
			\hline
			&&\\[-0.36cm]
			\multicolumn{3}{l}{\footnotesize Fuente:  INE. Encuesta Nacional de Salud Materno Infantil (Ensmi) 2008/2009.}\\[-.09cm]
			%			\multicolumn{3}{l}{\parbox{13cm}{\footnotesize \textbf{Nota:} Peso bajo es el menor a 18.4 onzas; peso normal es entre 18.5 y 24.9; el sobre peso al nacer es el mayor a 25.0 y menor 29.9 onzas; obesidad es mayor a 30.0 onzas.}}
		\end{tabular}\addtocounter{Cuadro}{1}
	\end{center}}
	


%%%%%%%%%%%%%%%%%%%%%%%%%



%%%%%%%%%%%%%%%%%%


\hoja{
	\normalsize
	{\Bold\color{color1!80!black}{Cuadro \theCuadro $\,-$Menores de 5 años por tipo de desnutrición; según área, región, categoría étnica y }}\\
	{\Bold\color{color1!80!black}{nivel de educación de la madre.}}\\
	{\Bold\color{color1!80!black}{República de Guatemala, año 2008/2009.}}\\
	{\color{color1!80!black}{(Porcentaje)}}\\[-0.3cm]
	\begin{center}\fontsize{3.0mm}{1.3em}\selectfont \setlength{\arrayrulewidth}{0.7pt}
		\begin{tabular}{x{2.0cm}cccccc}
			\hline &&&&&&\\[-0.5cm]  
			\multicolumn{1}{x{2.0cm}}{\multirow{3}{*}[1mm]{\textbf{Característica}}} &	\multicolumn{2}{x{3.2cm}}{\textbf{Desnutrición crónica}}&\multicolumn{2}{x{3.2cm}}{\Bold{Desnutrición aguda}}&\multicolumn{2}{x{3.2cm}}{\Bold{Desnutrición global}}\\[-0.05cm]
			\multicolumn{1}{x{2.0cm}}{ } &	\multicolumn{2}{x{3.2cm}}{\textbf{(Talla para la edad)}}&\multicolumn{2}{x{3.2cm}}{\Bold{(Peso para la talla)}}&\multicolumn{2}{x{3.2cm}}{\Bold{(Peso para la edad)}}\\[0.05cm]\cline{2-7}
			\multicolumn{1}{x{2.0cm}}{ } &	\multicolumn{1}{x{1.5cm}}{\textbf{Severa}}&\multicolumn{1}{x{1.5cm}}{\Bold{Total}}&	\multicolumn{1}{x{1.5cm}}{\textbf{Severa}}&\multicolumn{1}{x{1.5cm}}{\Bold{Total}}&	\multicolumn{1}{x{1.5cm}}{\textbf{Severa}}&\multicolumn{1}{x{1.5cm}}{\Bold{Total}}\\[0.05cm]
			\rowcolor{color1!40!white} \multicolumn{1}{l}{\Bold{	Total	 }}& 	21.2	 & 	49.8	 & 	0.5	 & 	1.4	 & 	2.1	 & 	13.1	 \\ 
			\rowcolor{color1!20!white} \multicolumn{1}{l}{\Bold{	Área geográfica	 }}& 		 & 		 & 		 & 		 & 		 & 		 \\ 
			\multicolumn{1}{l}{	Urbana	 }& 	11.6	 & 	34.3	 & 	0.4	 & 	1	 & 	1.1	 & 	8.2	 \\ 
			\rowcolor{color1!5!white}\multicolumn{1}{l}{	Rural	 }& 	26.7	 & 	58.6	 & 	0.6	 & 	1.6	 & 	2.6	 & 	15.9	 \\ 
			\rowcolor{color1!20!white} \multicolumn{1}{l}{\Bold{	Departamento	 }}& 		 & 		 & 		 & 		 & 		 & 		 \\ 
			\multicolumn{1}{l}{	Guatemala	 }& 	7.9	 & 	26.3	 & 	0.4	 & 	1.4	 & 	1.2	 & 	7.3	 \\ 
			\rowcolor{color1!5!white}\multicolumn{1}{l}{	El Progreso	 }& 	9.2	 & 	25.3	 & 	1	 & 	1.7	 & 	1.3	 & 	8	 \\ 
			\multicolumn{1}{l}{	Sacatepéquez	 }& 	17.7	 & 	51.4	 & 	1.4	 & 	1.6	 & 	0.8	 & 	8.5	 \\ 
			\rowcolor{color1!5!white}\multicolumn{1}{l}{	Chimaltenango	 }& 	23.8	 & 	61.2	 & 	0.8	 & 	1.2	 & 	3.3	 & 	14.5	 \\ 
			\multicolumn{1}{l}{	Escuintla	 }& 	10.3	 & 	32.4	 & 	0.2	 & 	0.8	 & 	0.5	 & 	10.2	 \\ 
			\rowcolor{color1!5!white}\multicolumn{1}{l}{	Santa Rosa	 }& 	10.1	 & 	28.9	 & 	0.8	 & 	1.7	 & 	0.6	 & 	7.7	 \\ 
			\multicolumn{1}{l}{	Sololá	 }& 	36.4	 & 	72.3	 & 	1	 & 	1	 & 	2.6	 & 	17.3	 \\ 
			\rowcolor{color1!5!white}\multicolumn{1}{l}{	Totonicapán	 }& 	42.8	 & 	82.2	 & 	0.5	 & 	0.9	 & 	3.1	 & 	24.5	 \\ 
			\multicolumn{1}{l}{	Quetzaltenango	 }& 	13.1	 & 	43.1	 & 	0.6	 & 	1.5	 & 	1	 & 	10	 \\ 
			\rowcolor{color1!5!white}\multicolumn{1}{l}{	Suchitepéquez	 }& 	13.3	 & 	43.5	 & 	1	 & 	2.3	 & 	1.7	 & 	12.5	 \\ 
			\multicolumn{1}{l}{	Retalhuleu	 }& 	10.5	 & 	34.6	 & 	0	 & 	2.3	 & 	2.5	 & 	11.5	 \\ 
			\rowcolor{color1!5!white}\multicolumn{1}{l}{	San Marcos	 }& 	21.5	 & 	53.5	 & 	1.8	 & 	2.9	 & 	1	 & 	14.4	 \\ 
			\multicolumn{1}{l}{	Huehuetenango	 }& 	36.5	 & 	69.5	 & 	0.1	 & 	1	 & 	4.2	 & 	20.8	 \\ 
			\rowcolor{color1!5!white}\multicolumn{1}{l}{	Quiché	 }& 	39.4	 & 	72.2	 & 	0.6	 & 	1	 & 	4.5	 & 	21.5	 \\ 
			\multicolumn{1}{l}{	Baja Verapaz	 }& 	29	 & 	59.4	 & 	0	 & 	1.6	 & 	3	 & 	14.9	 \\ 
			\rowcolor{color1!5!white}\multicolumn{1}{l}{	Alta Verapaz	 }& 	24.6	 & 	59.4	 & 	0.1	 & 	1.1	 & 	1.3	 & 	9.3	 \\ 
			\multicolumn{1}{l}{	Petén	 }& 	13.2	 & 	41.9	 & 	0.4	 & 	1	 & 	0.6	 & 	9	 \\ 
			\rowcolor{color1!5!white}\multicolumn{1}{l}{	Izabal	 }& 	12.3	 & 	40.4	 & 	0.4	 & 	2.8	 & 	3.5	 & 	13.2	 \\ 
			\multicolumn{1}{l}{	Zacapa	 }& 	21.3	 & 	45.9	 & 	0	 & 	0.4	 & 	3.7	 & 	16	 \\ 
			\rowcolor{color1!5!white}\multicolumn{1}{l}{	Chiquimula	 }& 	29.1	 & 	61.8	 & 	0.4	 & 	1.2	 & 	3.7	 & 	16.9	 \\ 
			\multicolumn{1}{l}{	Jalapa	 }& 	22.9	 & 	49.3	 & 	0	 & 	0.2	 & 	1.4	 & 	11.6	 \\ 
			\rowcolor{color1!5!white}\multicolumn{1}{l}{	Jutiapa	 }& 	14	 & 	36.8	 & 	0.5	 & 	1.8	 & 	0.8	 & 	10.5	 \\ 
			\rowcolor{color1!20!white} \multicolumn{1}{l}{\Bold{	Categoría étnica	 }}& 		 & 		 & 		 & 		 & 		 & 		 \\ 
			\multicolumn{1}{l}{	Indígena	 }& 	31.3	 & 	65.9	 & 	0.5	 & 	1.3	 & 	3	 & 	16.8	 \\ 
			\rowcolor{color1!5!white}\multicolumn{1}{l}{	Ladino	 }& 	12.7	 & 	36.2	 & 	0.5	 & 	1.5	 & 	1.3	 & 	10.1	 \\ 
			\rowcolor{color1!20!white} \multicolumn{1}{l}{\Bold{	Nivel de educación	 }}& 		 & 		 & 		 & 		 & 		 & 		 \\ 
			\multicolumn{1}{l}{	Sin educación	 }& 	35.9	 & 	69.3	 & 	0.8	 & 	1.6	 & 	3.5	 & 	19.9	 \\ 
			\rowcolor{color1!5!white}\multicolumn{1}{l}{	Primaria	 }& 	19.1	 & 	50.3	 & 	0.5	 & 	1.4	 & 	1.8	 & 	12.6	 \\ 
			\multicolumn{1}{l}{	Secundaria	 }& 	5.7	 & 	21.2	 & 	0.2	 & 	1.1	 & 	0.8	 & 	5.1	 \\ 
			\rowcolor{color1!5!white}\multicolumn{1}{l}{	Superior	 }& 	3.7	 & 	14.1	 & 	0.6	 & 	0.6	 & 	0.5	 & 	2.1	 \\[0.05cm]
			\hline
			&&&&&&\\[-0.36cm]
			\multicolumn{7}{l}{\footnotesize Fuente:  INE. Encuesta Nacional de Salud Materno Infantil (Ensmi) 2008/2009.}\\[.1cm]
			\multicolumn{7}{l}{\parbox{13cm}{\footnotesize \textbf{Notas:} Se denomina desnutrición severa cuando los niños están 3 desviaciones estándar o más por debajo de la media, de acuerdo a la tabla de medidas de la OMS. }}\\[.3cm]
			\multicolumn{7}{l}{\parbox{13cm}{\footnotesize Se denomina desnutrición total cuando los niños están dos desviaciones estándar o más por debajo de la media. Incluye a los niños que están 3 desviaciones estándar o más por debajo de la media.}}\\
		\end{tabular}\addtocounter{Cuadro}{1}
	\end{center}}
	


%%%%%%%%%%%%%%%%%%


\hoja{
	\normalsize
	{\Bold\color{color1!80!black}{Cuadro \theCuadro $\,-$Niños y niñas de 6 a 59 meses con anemia; según características varias. }}\\
	%	{\Bold\color{color1!80!black}{nivel de educación de la madre.}}\\
	{\Bold\color{color1!80!black}{República de Guatemala, año 2008/2009.}}\\
	{\color{color1!80!black}{(Porcentaje)}}\\[-0.3cm]
	\begin{center}\fontsize{3.0mm}{1.3em}\selectfont \setlength{\arrayrulewidth}{0.7pt}
		\begin{tabular}{x{4.0cm}c}
			\hline &\\[-0.5cm]  
			\multicolumn{1}{x{2.0cm}}{\textbf{Característica}} &	\multicolumn{1}{x{5cm}}{\textbf{Niños menores de 5 años}}\\[-0.05cm]\hline
			%			\multicolumn{1}{x{2.0cm}}{ } &	\multicolumn{2}{x{3.2cm}}{\textbf{(Talla para la edad)}}&\multicolumn{2}{x{3.2cm}}{\Bold{(Peso para la talla)}}&\multicolumn{2}{x{3.2cm}}{\Bold{(Peso para la edad)}}\\[0.05cm]\cline{2-7}
			%			\multicolumn{1}{x{2.0cm}}{ } &	\multicolumn{1}{x{1.5cm}}{\textbf{Severa}}&\multicolumn{1}{x{1.5cm}}{\Bold{Total}}&	\multicolumn{1}{x{1.5cm}}{\textbf{Severa}}&\multicolumn{1}{x{1.5cm}}{\Bold{Total}}&	\multicolumn{1}{x{1.5cm}}{\textbf{Severa}}&\multicolumn{1}{x{1.5cm}}{\Bold{Total}}\\[0.05cm]
			\rowcolor{color1!40!white} \multicolumn{1}{l}{\Bold{	Total República	 }}& 	 47.7 	 \\ 
			\rowcolor{color1!20!white} \multicolumn{1}{l}{\Bold{	Área geográfica	 }}& 		 \\ 
			\multicolumn{1}{l}{	Urbana	 }& 	 46.2 	 \\ 
			\rowcolor{color1!5!white}\multicolumn{1}{l}{	Rural	 }& 	 48.6 	 \\ 
			\rowcolor{color1!20!white} \multicolumn{1}{l}{\Bold{	Departamento	 }}& 		 \\ 
			\multicolumn{1}{l}{	Guatemala	 }& 	 40.7 	 \\ 
			\rowcolor{color1!5!white}\multicolumn{1}{l}{	El Progreso	 }& 	 37.8 	 \\ 
			\multicolumn{1}{l}{	Sacatepéquez	 }& 	 54.2 	 \\ 
			\rowcolor{color1!5!white}\multicolumn{1}{l}{	Chimaltenango	 }& 	 53.5 	 \\ 
			\multicolumn{1}{l}{	Escuintla	 }& 	 50.2 	 \\ 
			\rowcolor{color1!5!white}\multicolumn{1}{l}{	Santa Rosa	 }& 	 51.4 	 \\ 
			\multicolumn{1}{l}{	Sololá	 }& 	 56.1 	 \\ 
			\rowcolor{color1!5!white}\multicolumn{1}{l}{	Totonicapán	 }& 	 62.2 	 \\ 
			\multicolumn{1}{l}{	Quetzaltenango	 }& 	 40.2 	 \\ 
			\rowcolor{color1!5!white}\multicolumn{1}{l}{	Suchitepéquez	 }& 	 37.7 	 \\ 
			\multicolumn{1}{l}{	Retalhuleu	 }& 	 45.3 	 \\ 
			\rowcolor{color1!5!white}\multicolumn{1}{l}{	San Marcos	 }& 	 52.6 	 \\ 
			\multicolumn{1}{l}{	Huehuetenango	 }& 	 47.7 	 \\ 
			\rowcolor{color1!5!white}\multicolumn{1}{l}{	Quiché	 }& 	 47.4 	 \\ 
			\multicolumn{1}{l}{	Baja Verapaz	 }& 	 49.8 	 \\ 
			\rowcolor{color1!5!white}\multicolumn{1}{l}{	Alta Verapaz	 }& 	 46.1 	 \\ 
			\multicolumn{1}{l}{	Petén	 }& 	 48.5 	 \\ 
			\rowcolor{color1!5!white}\multicolumn{1}{l}{	Izabal	 }& 	 53.0 	 \\ 
			\multicolumn{1}{l}{	Zacapa	 }& 	 53.7 	 \\ 
			\rowcolor{color1!5!white}\multicolumn{1}{l}{	Chiquimula	 }& 	 55.5 	 \\ 
			\multicolumn{1}{l}{	Jalapa	 }& 	 43.9 	 \\ 
			\rowcolor{color1!5!white}\multicolumn{1}{l}{	Jutiapa	 }& 	 50.3 	 \\ 
			\rowcolor{color1!20!white} \multicolumn{1}{l}{\Bold{	Categoría étnica	 }}& 		 \\ 
			\multicolumn{1}{l}{	Indígena	 }& 	 49.5 	 \\ 
			\rowcolor{color1!5!white}\multicolumn{1}{l}{	Ladino	 }& 	 46.3 	 \\ 
			\rowcolor{color1!20!white} \multicolumn{1}{l}{\Bold{	Nivel de educación	 }}& 		 \\ 
			\multicolumn{1}{l}{	Sin educación	 }& 	 48.3 	 \\ 
			\rowcolor{color1!5!white}\multicolumn{1}{l}{	Primaria	 }& 	 49.2 	 \\ 
			\multicolumn{1}{l}{	Secundaria	 }& 	 44.0 	 \\ 
			\rowcolor{color1!5!white}\multicolumn{1}{l}{	Superior	 }& 	 36.0 	 \\ 
			[0.05cm]
			\hline
			&\\[-0.36cm]\end{tabular}\addtocounter{Cuadro}{1}
	\end{center}
	{\footnotesize Fuente:  INE. Encuesta Nacional de Salud Materno Infantil (Ensmi) 2008/2009.}\\[.1cm]
	{\parbox{13cm}{\footnotesize \textbf{Notas:} Se denomina desnutrición severa cuando los niños están 3 desviaciones estándar o más por debajo de la media, de acuerdo a la tabla de medidas de la OMS. }}\\[.3cm]
	{\parbox{13cm}{\footnotesize Se denomina desnutrición total cuando los niños están dos desviaciones estándar o más por debajo de la media. Incluye a los niños que están 3 desviaciones estándar o más por debajo de la media.}}\\
}

%%%%%%%%%%%%%%%%%%


\hoja{
	\normalsize
	{\Bold\color{color1!80!black}{Cuadro \theCuadro $\,-$ Distribución de los niños recién nacidos; según peso reportado por la madre. }}\\
	%	{\Bold\color{color1!80!black}{nivel de educación de la madre.}}\\
	{\Bold\color{color1!80!black}{República de Guatemala, año 2008/2009.}}\\
	{\color{color1!80!black}{(Porcentaje)}}\\[-0.3cm]
	\begin{center}\fontsize{3.0mm}{1.3em}\selectfont \setlength{\arrayrulewidth}{0.7pt}
		\begin{tabular}{x{4.0cm}ccc}
			\hline &\\[-0.5cm]  
			\multicolumn{1}{x{2.0cm}}{\textbf{Característica}} &	\multicolumn{1}{x{2.5cm}}{\textbf{Menos de 2.5 kilos}}&\multicolumn{1}{x{2cm}}{\textbf{2.5 kilos o más}}&\multicolumn{1}{x{2cm}}{\textbf{No sabe}}\\[0.05cm]\hline
			%			\multicolumn{1}{x{2.0cm}}{ } &	\multicolumn{2}{x{3.2cm}}{\textbf{(Talla para la edad)}}&\multicolumn{2}{x{3.2cm}}{\Bold{(Peso para la talla)}}&\multicolumn{2}{x{3.2cm}}{\Bold{(Peso para la edad)}}\\[0.05cm]\cline{2-7}
			%			\multicolumn{1}{x{2.0cm}}{ } &	\multicolumn{1}{x{1.5cm}}{\textbf{Severa}}&\multicolumn{1}{x{1.5cm}}{\Bold{Total}}&	\multicolumn{1}{x{1.5cm}}{\textbf{Severa}}&\multicolumn{1}{x{1.5cm}}{\Bold{Total}}&	\multicolumn{1}{x{1.5cm}}{\textbf{Severa}}&\multicolumn{1}{x{1.5cm}}{\Bold{Total}}\\[0.05cm]
			\rowcolor{color1!40!white} \multicolumn{1}{l}{\Bold{	Total	 }}& 	11.4	 & 	88.1	 & 	0.5	 \\ 
			\rowcolor{color1!20!white} \multicolumn{1}{l}{\Bold{	Área geográfica	 }}& 		 & 		 & 		 \\ 
			\multicolumn{1}{l}{	Urbana	 }& 	 12.6 	 & 	 86.9 	 & 	 0.5 	 \\ 
			\rowcolor{color1!5!white}\multicolumn{1}{l}{	Rural	 }& 	 10.6 	 & 	 88.9 	 & 	 0.5 	 \\ 
			\rowcolor{color1!20!white} \multicolumn{1}{l}{\Bold{	Departamentos	 }}& 		 & 		 & 		 \\ 
			\multicolumn{1}{l}{	Guatemala	 }& 	 13.7 	 & 	 85.7 	 & 	 0.6 	 \\ 
			\rowcolor{color1!5!white}\multicolumn{1}{l}{	El Progreso	 }& 	 13.7 	 & 	 85.1 	 & 	 1.2 	 \\ 
			\multicolumn{1}{l}{	Sacatepéquez	 }& 	 14.6 	 & 	 85.4 	 & 	 -   	 \\ 
			\rowcolor{color1!5!white}\multicolumn{1}{l}{	Chimaltenango	 }& 	 9.1 	 & 	 90.9 	 & 	 -   	 \\ 
			\multicolumn{1}{l}{	Escuintla	 }& 	 10.3 	 & 	 89.2 	 & 	 0.6 	 \\ 
			\rowcolor{color1!5!white}\multicolumn{1}{l}{	Santa Rosa	 }& 	 6.8 	 & 	 92.4 	 & 	 0.8 	 \\ 
			\multicolumn{1}{l}{	Sololá	 }& 	 10.8 	 & 	 88.4 	 & 	 0.8 	 \\ 
			\rowcolor{color1!5!white}\multicolumn{1}{l}{	Totonicapán	 }& 	 11.7 	 & 	 88.2 	 & 	 0.2 	 \\ 
			\multicolumn{1}{l}{	Quetzaltenango	 }& 	 14.7 	 & 	 85.3 	 & 	 -   	 \\ 
			\rowcolor{color1!5!white}\multicolumn{1}{l}{	Suchitepéquez	 }& 	 5.6 	 & 	 94.3 	 & 	 0.2 	 \\ 
			\multicolumn{1}{l}{	Retalhuleu	 }& 	 6.7 	 & 	 93.3 	 & 	 -   	 \\ 
			\rowcolor{color1!5!white}\multicolumn{1}{l}{	San Marcos	 }& 	 9.4 	 & 	 90.0 	 & 	 0.6 	 \\ 
			\multicolumn{1}{l}{	Huehuetenango	 }& 	 10.1 	 & 	 89.1 	 & 	 0.8 	 \\ 
			\rowcolor{color1!5!white}\multicolumn{1}{l}{	Quiché	 }& 	 11.8 	 & 	 86.7 	 & 	 1.5 	 \\ 
			\multicolumn{1}{l}{	Baja Verapaz	 }& 	 18.7 	 & 	 80.7 	 & 	 0.6 	 \\ 
			\rowcolor{color1!5!white}\multicolumn{1}{l}{	Alta Verapaz	 }& 	 12.9 	 & 	 86.9 	 & 	 0.1 	 \\ 
			\multicolumn{1}{l}{	Petén	 }& 	 9.7 	 & 	 89.6 	 & 	 0.8 	 \\ 
			\rowcolor{color1!5!white}\multicolumn{1}{l}{	Izabal	 }& 	 7.5 	 & 	 92.5 	 & 	 -   	 \\ 
			\multicolumn{1}{l}{	Zacapa	 }& 	 11.0 	 & 	 89.0 	 & 	 -   	 \\ 
			\rowcolor{color1!5!white}\multicolumn{1}{l}{	Chiquimula	 }& 	 15.0 	 & 	 84.7 	 & 	 0.3 	 \\ 
			\multicolumn{1}{l}{	Jalapa	 }& 	 14.6 	 & 	 85.4 	 & 	 -   	 \\ 
			\rowcolor{color1!5!white}\multicolumn{1}{l}{	Jutiapa	 }& 	 9.6 	 & 	 90.4 	 & 	 -   	 \\ 
			\rowcolor{color1!20!white} \multicolumn{1}{l}{\Bold{	Categoría étnica	 }}& 		 & 		 & 		 \\ 
			\multicolumn{1}{l}{	Indígena	 }& 	 11.7 	 & 	 87.7 	 & 	 0.6 	 \\ 
			\rowcolor{color1!5!white}\multicolumn{1}{l}{	Ladino	 }& 	 11.2 	 & 	 88.4 	 & 	 0.4 	 \\ 
			\rowcolor{color1!20!white} \multicolumn{1}{l}{\Bold{	Nivel de educación	 }}& 		 & 		 & 		 \\ 
			\multicolumn{1}{l}{	Sin educación	 }& 	 12.1 	 & 	 87.1 	 & 	 0.9 	 \\ 
			\rowcolor{color1!5!white}\multicolumn{1}{l}{	Primaria	 }& 	 10.7 	 & 	 88.9 	 & 	 0.4 	 \\ 
			\multicolumn{1}{l}{	Secundaria	 }& 	 12.6 	 & 	 87.0 	 & 	 0.4 	 \\ 
			\rowcolor{color1!5!white}\multicolumn{1}{l}{	Superior	 }& 	 10.2 	 & 	 89.8 	 & 		 \\ 
			[0.05cm]
			\hline
			&&&\\[-0.36cm]\end{tabular}\addtocounter{Cuadro}{1}
	\end{center}
	{\footnotesize Fuente:  INE. Encuesta Nacional de Salud Materno Infantil (Ensmi) 2008/2009.}\\[.1cm]
	%	{\parbox{13cm}{\footnotesize \textbf{Notas:} Se denomina desnutrición severa cuando los niños están 3 desviaciones estándar o más por debajo de la media, de acuerdo a la tabla de medidas de la OMS. }}\\[.3cm]
	%	{\parbox{13cm}{\footnotesize Se denomina desnutrición total cuando los niños están dos desviaciones estándar o más por debajo de la media. Incluye a los niños que están 3 desviaciones estándar o más por debajo de la media.}}\\
}


